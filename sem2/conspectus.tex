\documentclass[12pt, fleqn]{book}
% Эта строка — комментарий, она не будет показана в выходном файле
\usepackage{cmap}
\usepackage{ucs}
\usepackage[utf8x]{inputenc} % Включаем поддержку UTF8
\usepackage[english, russian]{babel}  % Включаем пакет для поддержки русского языка
\usepackage{amsmath}
\usepackage{amssymb}
\usepackage{amsthm}
\usepackage{amsfonts}
\usepackage{array}
\usepackage[makeroom]{cancel}
\usepackage{enumerate}

\usepackage{tikz}
%\DeclareMathSizes{10}{10}{10}{10}
\title{Конспект лекций по МатАн}
\date{03.02.2025}
\author{Черепанов Илья}

\begin{document}
	\maketitle
	\tableofcontents{}
	\chapter{Интегралы}
	\section{Первообразная и неопределенный интеграл}
	\subsection{Определение}
	$F(x), f(x)$ -- определены на $X$\\
	$F(x)$ -- первообразная для $f(x)$, если $F^\prime(x)=f(x)$\\
	\textit{Пример:}\\
	$x^2$ -- первообразная для $2x$\\
	$x^2+5$ -- первоообразная для $2x$
	\paragraph{Теорема} $F_1(x), F_2(x)$ -- первообразные для $f(x) \Rightarrow F_1(x)=F_2(x)+c$ $(c=const)$\\
	\textit{Доказательство:}
	\begin{multline*}
		\left(F_1(x) - F_2(x)\right)^\prime = F_1^\prime(x) - F_2^\prime(x)=f(x)-f(x)=0\\
		\Rightarrow F_1(x)-F_2(x) = c\\
		\blacksquare\\
	\end{multline*}
	\paragraph{Неопределенный интеграл от функции $f(x)$ на $X$} -- совокупность всех первообразных $f(x)$ на $X$.
	\begin{multline*}
		\int f(x)dx=F(x)+c\\
		c\in \mathbb{R}\\
	\end{multline*}
	\subsection{Свойства неопределенных интегралов}
	\begin{enumerate}[1)]
		\item $\begin{gathered}
		 	\left(\int f(x)dx\right)^\prime=f(x)
		 \end{gathered}$
		\item 
		$\begin{gathered}
			\int f(x)dx=f(x) + c
		\end{gathered}$
		\item 
	$	\begin{gathered}
			\int kf(x)dx=k\int f(x)dx
		\end{gathered}$\\
		\textit{Доказательство:}
		\begin{multline*}
			\left(\int kf(x)dx\right)^\prime=kf(x)\\
			\left(k\int f(x)dx\right)^\prime=k\left(\int f(x) dx\right)^\prime=kf(x)\\
			\blacksquare\\
		\end{multline*}
		\item $\begin{gathered}
			\int \left(f_1(x) + f_2(x)\right)dx = \int f_1(x) dx + \int f_2(x) dx
		\end{gathered}$
	\end{enumerate}
	\subsection{Таблица интегралов}
	\begin{enumerate}[1)]
		\item $\begin{gathered}
			\int x^mdx=\frac{x^{m+1}}{m+1}+c, \space \space m \neq -1
		\end{gathered}$
		\textit{$ $}
	\end{enumerate}
\end{document}
