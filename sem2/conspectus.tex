\documentclass[12pt, fleqn]{book}
% Эта строка — комментарий, она не будет показана в выходном файле
\usepackage{cmap}
\usepackage{ucs}
\usepackage[utf8x]{inputenc} % Включаем поддержку UTF8
\usepackage[english, russian]{babel}  % Включаем пакет для поддержки русского языка
\usepackage{amsmath}
\usepackage{amssymb}
\usepackage{amsthm}
\usepackage{amsfonts}
\usepackage{array}
\usepackage[makeroom]{cancel}
\usepackage{enumerate}

\usepackage{tikz}
%\DeclareMathSizes{10}{10}{10}{10}
\title{Конспект лекций по МатАн}
\date{03.02.2025}
\author{Черепанов Илья}

\begin{document}
	\maketitle
	\tableofcontents{}
	\chapter{Интегралы}
	\section{Первообразная и неопределенный интеграл}
	\subsection{Определение}
	$F(x), f(x)$ -- определены на $X$\\
	$F(x)$ -- первообразная для $f(x)$, если $F^\prime(x)=f(x)$\\
	\textit{Пример:}\\
	$x^2$ -- первообразная для $2x$\\
	$x^2+5$ -- первоообразная для $2x$
	\paragraph{Теорема} $F_1(x), F_2(x)$ -- первообразные для $f(x) \Rightarrow F_1(x)=F_2(x)+C$ $(C=const)$\\
	\textit{Доказательство:}
	\begin{multline*}
		\left(F_1(x) - F_2(x)\right)^\prime = F_1^\prime(x) - F_2^\prime(x)=f(x)-f(x)=0\\
		\Rightarrow F_1(x)-F_2(x) = C\\
		\blacksquare\\
	\end{multline*}
	\paragraph{Неопределенный интеграл от функции $f(x)$ на $X$} -- совокупность всех первообразных $f(x)$ на $X$.
	\begin{multline*}
		\int f(x)dx=F(x)+C\\
		C\in \mathbb{R}\\
	\end{multline*}
	\subsection{Свойства неопределенных интегралов}
	\begin{enumerate}[1)]
		\item $\begin{gathered}
		 	\left(\int f(x)dx\right)^\prime=f(x)
		 \end{gathered}$
		\item 
		$\begin{gathered}
			\int f(x)dx=f(x) + C
		\end{gathered}$
		\item 
	$	\begin{gathered}
			\int kf(x)dx=k\int f(x)dx
		\end{gathered}$\\
		\textit{Доказательство:}
		\begin{multline*}
			\left(\int kf(x)dx\right)^\prime=kf(x)\\
			\left(k\int f(x)dx\right)^\prime=k\left(\int f(x) dx\right)^\prime=kf(x)\\
			\blacksquare\\
		\end{multline*}
		\item $\begin{gathered}
			\int \left(f_1(x) + f_2(x)\right)dx = \int f_1(x) dx + \int f_2(x) dx
		\end{gathered}$
	\end{enumerate}
	\subsection{Таблица интегралов}
	\begin{enumerate}[1)]
		\item $\begin{gathered}
			\int x^mdx=\frac{x^{m+1}}{m+1}+C, \space \space m \neq -1
		\end{gathered}$
		\item $\begin{gathered}
			\int \frac{1}{x}dx=\ln\left| x\right| + C
		\end{gathered}$
		\item $\begin{gathered}
			\int e^xdx=e^x+C
		\end{gathered}$
		\item $\begin{gathered}
			\int a^xdx=\frac{a^x}{\ln a}+C
		\end{gathered}$
		\item $\begin{gathered}
			\int \sin xdx=-\cos x + C
		\end{gathered}$
		\item $\begin{gathered}
			\int \cos x dx = \sin x + C
		\end{gathered}$
		\item $\begin{gathered}
			\int \frac{1}{\cos^2x}dx=\tg x + C
		\end{gathered}$
		\item $\begin{gathered}
			\int \frac{1}{\sin^2x}dx=-\ctg x + C
		\end{gathered}$
		\item $\begin{gathered}
			\int \frac{dx}{\sqrt{1-x^2}}=\arcsin x + C
		\end{gathered}$
		\item $\begin{gathered}
			\int \frac{dx}{1+x^2}=\arctg x + C
		\end{gathered}$
		\item $\begin{gathered}
			\int \frac{dx}{x^2-a^2}=\frac{1}{2a}\ln\left|\frac{x-a}{x+a} \right| +C
		\end{gathered}$
		\item $\begin{gathered}
			\int \frac{dx}{\sqrt{x^2\pm a^2}}=\ln \left| x+\sqrt{x^2\pm a^2}\right| + C
		\end{gathered}$
	\end{enumerate}
	\subsection{Инвариативность формул интегрирования}
	\paragraph{Теорема}
	$u(x)$ -- непрерывна и дифференцируема на $X$. \\$\begin{gathered}
		\int f(x)dx=F(x)+C
	\end{gathered}$\\
	Область значений $u(x)$ совпадает с областью определения $f(x) \Rightarrow$.\\
	$\Rightarrow\begin{gathered}
		\int f\left(u\left(x\right)\right)u^\prime\left(x\right)dx=F\left(u\left(x\right)\right)+C
	\end{gathered}$\\\\
	\textit{Доказательство:}\\
	\begin{multline*}
		\left.
		\begin{aligned}
			&\text{Производная левой части: }& f\left(u\left(x\right)\right)u^\prime\left(x\right)&\\
			&\text{Производная правой части: }& F_u^\prime\left(u\right)u^\prime\left(x\right)=f\left(u\left(x\right)\right)u^\prime\left(x\right)&\\
		\end{aligned}
		\right\} \Rightarrow \text{равенство верно}\\
		\blacksquare\\
	\end{multline*}
	\subsubsection{Метод внесения под знак дифференциала}
	\begin{equation*}
		\int f(x)dx=F(x)+c \Rightarrow \int f(u)du=F(u)+C
	\end{equation*}
	\begin{enumerate}[1.]
		\item $\begin{gathered}
			dx=\frac{1}{a}d\left(ax+b\right)
		\end{gathered}$\\\\
		\textit{Примеры:}
		\begin{enumerate}[1)]
			\item $\begin{gathered}
				\int e^{2x}dx=\frac{1}{2}\int e^{2x}d2x=\frac{1}{2}\int e^udu=\frac{1}{2}e^{2x}+C
			\end{gathered}$
			\item $\begin{gathered}
				\int \frac{dx}{3x+1}=\frac{1}{3}\int \frac{d3x}{3x+1}=\frac{1}{3}\int \frac{d(3x+1)}{3x+1}=\frac{1}{3}\ln\left|3x+1 \right|+C
			\end{gathered}$
			\item $\begin{gathered}
				\int \frac{dx}{\sqrt[4]{1-2x}}=-\frac{1}{2}\int \frac{d(-2x+1)}{\sqrt[4]{1-2x}}=-\frac{2}{3}\left(1-2x\right)^\frac{3}{4}+C
			\end{gathered}$
		\end{enumerate}
		\item $\begin{gathered}
			\frac{dx}{x}=d\ln x \left\lgroup \ln \left| x\right|\right\rgroup
		\end{gathered}$\\\\
		\textit{Примеры:}
		\begin{enumerate}[1)]
			\item $\begin{gathered}
				\int \frac{dx}{x \ln x}=\int \frac{d \ln x}{\ln x}=\ln\left| \ln x\right| + C
			\end{gathered}$
			\item $\begin{gathered}
				\int \frac{dx}{x\sqrt{2-5\ln x}}=\int \frac{d\ln x}{\sqrt{2-5\ln x}}=-\frac{1}{5}\int \frac{d\left(-5\ln x+2\right)}{\sqrt{2-5\ln x}}=-\frac{2}{5}\sqrt{2-5\ln x} +C
			\end{gathered}$
			\item $\begin{gathered}
				\int \frac{dx}{x\left(3\ln x + 2\right)^2}=-\frac{1}{3\left(3\ln x + 2\right)} + C
			\end{gathered}$
		\end{enumerate}
	\end{enumerate}
\end{document}
