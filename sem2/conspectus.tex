\documentclass[12pt, fleqn]{book}
% Эта строка — комментарий, она не будет показана в выходном файле
\usepackage{cmap}
\usepackage{ucs}
\usepackage[utf8x]{inputenc} % Включаем поддержку UTF8
\usepackage[english, russian]{babel}  % Включаем пакет для поддержки русского языка
\usepackage{amsmath}
\usepackage{amssymb}
\usepackage{amsthm}
\usepackage{amsfonts}
\usepackage{array}
\usepackage[makeroom]{cancel}
\usepackage{enumerate}
\usepackage{mdframed}

\usepackage{tikz}
%\DeclareMathSizes{10}{10}{10}{10}
\title{Конспект лекций по МатАн}
\date{03.02.2025}
\author{Черепанов Илья}

\begin{document}
	\maketitle
	\tableofcontents{}
	\chapter{Интегралы}
	\section{Первообразная и неопределенный интеграл}
	\subsection{Определение}
	$F(x), f(x)$ -- определены на $X$\\
	$F(x)$ -- первообразная для $f(x)$, если $F^\prime(x)=f(x)$\\
	\textit{Пример:}\\
	$x^2$ -- первообразная для $2x$\\
	$x^2+5$ -- первоообразная для $2x$
	\theoremstyle{plain}
	\newtheorem*{Primitive Integral}{Теорема}
	\begin{Primitive Integral}
		$F_1(x), F_2(x)$ -- первообразные для $f(x) \Rightarrow F_1(x)=F_2(x)+C$ $(C=const)$
	\end{Primitive Integral}
	\begin{proof}
		\begin{multline*}
		\left(F_1(x) - F_2(x)\right)^\prime = F_1^\prime(x) - F_2^\prime(x)=f(x)-f(x)=0\\
		\Rightarrow F_1(x)-F_2(x) = C\\
		\end{multline*}
	\end{proof}
	\paragraph{Неопределенный интеграл от функции $f(x)$ на $X$} -- совокупность всех первообразных $f(x)$ на $X$.
	\begin{multline*}
		\int f(x)dx=F(x)+C\\
		C\in \mathbb{R}\\
	\end{multline*}
	\subsection{Свойства неопределенных интегралов}
	\begin{enumerate}[1)]
		\item $\begin{gathered}
		 	\left(\int f(x)dx\right)^\prime=f(x)
		 \end{gathered}$
		\item 
		$\begin{gathered}
			\int f(x)dx=f(x) + C
		\end{gathered}$
		\item 
	$	\begin{gathered}
			\int kf(x)dx=k\int f(x)dx
		\end{gathered}$
		\begin{proof}
			\begin{multline*}
				\left(\int kf(x)dx\right)^\prime=kf(x)\\
				\left(k\int f(x)dx\right)^\prime=k\left(\int f(x) dx\right)^\prime=kf(x)\\
			\end{multline*}
		\end{proof}
		\item $\begin{gathered}
			\int \left(f_1(x) + f_2(x)\right)dx = \int f_1(x) dx + \int f_2(x) dx
		\end{gathered}$
	\end{enumerate}
	\subsection{Таблица интегралов}
	\begin{enumerate}[1)]
		\item $\begin{gathered}
			\int x^mdx=\frac{x^{m+1}}{m+1}+C, \space \space m \neq -1
		\end{gathered}$
		\item $\begin{gathered}
			\int \frac{1}{x}dx=\ln\left| x\right| + C
		\end{gathered}$
		\item $\begin{gathered}
			\int e^xdx=e^x+C
		\end{gathered}$
		\item $\begin{gathered}
			\int a^xdx=\frac{a^x}{\ln a}+C
		\end{gathered}$
		\item $\begin{gathered}
			\int \sin xdx=-\cos x + C
		\end{gathered}$
		\item $\begin{gathered}
			\int \cos x dx = \sin x + C
		\end{gathered}$
		\item $\begin{gathered}
			\int \frac{1}{\cos^2x}dx=\tg x + C
		\end{gathered}$
		\item $\begin{gathered}
			\int \frac{1}{\sin^2x}dx=-\ctg x + C
		\end{gathered}$
		\item $\begin{gathered}
			\int \frac{dx}{\sqrt{1-x^2}}=\arcsin x + C
		\end{gathered}$
		\item $\begin{gathered}
			\int \frac{dx}{1+x^2}=\arctg x + C
		\end{gathered}$
		\item $\begin{gathered}
			\int \frac{dx}{x^2-a^2}=\frac{1}{2a}\ln\left|\frac{x-a}{x+a} \right| +C
		\end{gathered}$
		\item $\begin{gathered}
			\int \frac{dx}{\sqrt{x^2\pm a^2}}=\ln \left| x+\sqrt{x^2\pm a^2}\right| + C
		\end{gathered}$
	\end{enumerate}
	\subsection{Инвариативность формул интегрирования}
	\newtheorem*{Integral of a complex function}{Теорема}
	\begin{Integral of a complex function}
		$u(x)$ -- непрерывна и дифференцируема на $X$. \\$\begin{gathered}
			\int f(x)dx=F(x)+C
		\end{gathered}$\\
		Область значений $u(x)$ совпадает с областью определения $f(x) \Rightarrow$.\\
		$\Rightarrow\begin{gathered}
		\int f\left(u\left(x\right)\right)u^\prime\left(x\right)dx=F\left(u\left(x\right)\right)+C
		\end{gathered}$
	\end{Integral of a complex function}
	\begin{proof}
		\begin{multline*}
			\left.
			\begin{aligned}
				&\text{Производная левой части: }& f\left(u\left(x\right)\right)u^\prime\left(x\right)&\\
				&\text{Производная правой части: }& F_u^\prime\left(u\right)u^\prime\left(x\right)=f\left(u\left(x\right)\right)u^\prime\left(x\right)&\\
			\end{aligned}
			\right\} \Rightarrow \text{равенство верно}\\
		\end{multline*}
	\end{proof}
	\subsubsection{Метод внесения под знак дифференциала}
	\begin{equation*}
		\int f(x)dx=F(x)+c \Rightarrow \int f(u)du=F(u)+C
	\end{equation*}
	\begin{enumerate}[1.]
		\item \boxed{dx=\frac{d\left(ax+b\right)}{a}}\\\\
		\textit{Примеры:}
		\begin{enumerate}[1)]
			\item $\begin{gathered}
				\int e^{2x}dx=\frac{1}{2}\int e^{2x}d2x=\frac{1}{2}\int e^udu=\frac{1}{2}e^{2x}+C
			\end{gathered}$
			\item $\begin{gathered}
				\int \frac{dx}{3x+1}=\frac{1}{3}\int \frac{d3x}{3x+1}=\frac{1}{3}\int \frac{d(3x+1)}{3x+1}=\frac{1}{3}\ln\left|3x+1 \right|+C
			\end{gathered}$
			\item $\begin{gathered}
				\int \frac{dx}{\sqrt[4]{1-2x}}=-\frac{1}{2}\int \frac{d(-2x+1)}{\sqrt[4]{1-2x}}=-\frac{2}{3}\left(1-2x\right)^\frac{3}{4}+C
			\end{gathered}$
		\end{enumerate}
		\item \boxed{\frac{dx}{x}=d\ln x \left\lgroup \ln x \right\rgroup}\\\\
		\textit{Примеры:}
		\begin{enumerate}[1)]
			\item $\begin{gathered}
				\int \frac{dx}{x \ln x}=\int \frac{d \ln x}{\ln x}=\ln\left| \ln x\right| + C
			\end{gathered}$
			\item $\begin{gathered}
				\int \frac{dx}{x\sqrt{2-5\ln x}}=\int \frac{d\ln x}{\sqrt{2-5\ln x}}=-\frac{1}{5}\int \frac{d\left(-5\ln x+2\right)}{\sqrt{2-5\ln x}}=-\frac{2}{5}\sqrt{2-5\ln x} +C
			\end{gathered}$
			\item $\begin{gathered}
				\int \frac{dx}{x\left(3\ln x + 2\right)^2}=-\frac{1}{3\left(3\ln x + 2\right)} + C
			\end{gathered}$
		\end{enumerate}
		\item \boxed{xdx=\frac{dx^2}{2}}\\\\
		\textit{Примеры:}
		\begin{enumerate}[1)]
			\item $\begin{gathered}
				\int x e^{x^2}dx=\frac{1}{2}\int e^{x^2}dx^2=\frac{e^{x^2}}{2} + C
			\end{gathered}$
			\item $\begin{gathered}
				\int \frac{xdx}{\sqrt{2-3x^2}}=\frac{1}{2}\int \frac{dx^2}{\sqrt{2-3x^2}}=-\frac{1}{2\cdot3}\int \frac{d\left(-3x^2 +2\right)}{\sqrt{2-3x^2}}=-\frac{2}{6}\sqrt{2-3x^2} + C
			\end{gathered}$
			\item $\begin{gathered}
				\int \frac{xdx}{2x^2+5}=\frac{1}{2}\int \frac{dx^2}{2x^2+5}=\frac{1}{4}\int \frac{d\left(2x^2+5\right)}{2x^2+5}=\frac{1}{4}\ln \left(2x^2+5\right) + C
			\end{gathered}$
		\end{enumerate}
		\item \boxed{e^xdx=de^x}\\\\
		\textit{Примеры:}
		\begin{enumerate}[1)]
			\item $\begin{gathered}
				\int e^{e^x}e^xdx=\int e^{e^x}de^x=e^{e^x}+C
			\end{gathered}$
			\item $\begin{gathered}
				\int e^x \sqrt[3]{2e^x-1}dx=\int \sqrt[3]{2e^x-1}d\left(2e^x-1\right)=\frac{3}{8}\sqrt[3]{\left(2e^x-1\right)^4}+C
			\end{gathered}$
		\end{enumerate}
		\item \boxed{\cos xdx=d\sin x}  \space\space \boxed{\sin xdx=-d\cos x}\\\\
		\textit{Примеры:}
		\begin{enumerate}[1)]
			\item $\begin{gathered}
				\int \cos x e^{2\sin x+1}dx=\int e^{2\sin x+1}d\left(2\sin x+1\right)=\frac{1}{2}e^{2\sin x+1}+C
			\end{gathered}$
			\item $\begin{gathered}
				\int \frac{\sin xdx}{1+\cos^2x}=-\int\frac{d\cos x}{1+\cos^2x}=-\arctg\left(\cos x\right)+C
			\end{gathered}$
		\end{enumerate}
		\item \boxed{\frac{dx}{\sqrt{1-x^2}}=d\arcsin x}\\\\
		\textit{Примеры:}
		\begin{enumerate}[1)]
			\item $\begin{gathered}
				\int \frac{2^{\arcsin x}}{\sqrt{1-x^2}}dx=\int2^{\arcsin x}d\arcsin x=\frac{1}{\ln2}2^{\arcsin x}
			\end{gathered}$
		\end{enumerate}
	\end{enumerate}
	\subsection{Метод интегрирования по частям}
	\newtheorem*{Part Integration}{Теорема}
	\begin{Part Integration}
		$u(x), v(x)$ -- непрерывны и дифференцируемы на $X \Rightarrow$ $$
			\Rightarrow 
			\begin{aligned}
				\int &udv=uv-\int vdu&\\
				&\text{удав, увы, \space \space в аду}&\\
			\end{aligned}
	$$
	\end{Part Integration}
	\begin{proof}
		\begin{align*}
			(\text{левая часть})^\prime &= \left(\int uv^\prime dx\right)^\prime=uv^\prime\\
			(\text{правая часть})^\prime &= \left(uv\right)^\prime - \left(\int vu^\prime dx\right)^\prime=u^\prime v+uv^\prime-u^\prime v=uv^\prime
		\end{align*}
	\end{proof}
	\subsubsection{При вычислении}
	$u(x)$ -- функция, которая при дифференцировании упрощается\\\\
	\textit{Пример:}
	\begin{multline*}
		\int xe^xdx=xe^x-\int e^xdx=xe^x-e^x+C\\
		u=x;du=dx\\
		dv=e^xdx;v=\int e^xdx=e^x\\
	\end{multline*}
	\section{Определенный интеграл}
	\subsection{Определение интеграла по промежутку $[a;b]$}
	$f(x)$ -- определена на $[a;b]$ и ограничена.
	\begin{enumerate}
		\item $a=x_0<x_1<x_2<\dots<x_n=b$\\
		\begin{tikzpicture}
			\draw[->] (-0.5,0)--(4,0) node[right]{$x$};
			\draw[->] (0,-0.5)--(0,3) node[above]{$y$};
			\draw[line width=1pt] plot[domain=-0.5:2.5] (\x+1,{1+(\x-2)*(\x-2)*\x*\x});
				
				\foreach \x in {-0.5,-0.1,...,1.5}{
					\draw [dashed] (1+\x, 0)--(1+\x, {1+(\x-2)*(\x-2)*\x*\x});
				};
				\foreach \x in {0,1,...,5}{
					\draw ({0.5+4*\x/10}, -0.1) node[anchor=north]{$x_{\x}$} --({0.5+4*\x/10}, 0.1);
				};
				\foreach \x in {2.5,2.65,...,3.5}{
					\draw ({\x}, -0.1) --(\x, 0.1);
				};
				\draw (3.5, -0.1) node[anchor=north]{$x_{n}$} --(3.5, 0.1);
				\draw [dashed] (3.5, 0)--(3.5, 2.56);
		\end{tikzpicture}
		\item $\overline{x_i}\in \left(x_{i-1}; x_i\right), i = 1,2,\dots,n$
		\item $f\left(\overline{x_i}\right)$
		\item $\begin{gathered}
			\sum_{i=1}^{n}f\left(\overline{x_i}\right)\left(x_i-x_{i-1}\right)
		\end{gathered}$ -- интегральная сумма\\
		$x_i-x_{i-1}=\varDelta i$\\
		$\lambda = \max \varDelta i$
		\item $\begin{gathered}
			\lim_{\lambda\to0}\sum_{i=1}^{n}f\left(\overline{x_i}\right)\varDelta i
		\end{gathered}$\\
		Если этот предел существует и не зависит ни от способа разбиения промежутка $[a;b]$, ни от выбора точек $\overline{x_i}$, то он называется \textbf{\textit{определенным интегралом}} от функции $f(x)$ по промежутку $[a;b]$
	\end{enumerate}
	\newtheorem*{Existing Integral}{Теорема}
	\begin{Existing Integral}
		$f(x)$ -- кусочно непрерывна на $\begin{gathered}
			[a;b] \Rightarrow \exists \int_{a}^{b}f(x)dx
		\end{gathered}$
	\end{Existing Integral}
	\subsection{Свойства интеграла по промежутку}
	\begin{enumerate}
		\item $\begin{gathered}
			\int_{a}^{b}1dx=b-a
		\end{gathered}$
		\item $\begin{gathered}
			\int_{a}^{b}cf(x)dx=c\int_{a}^{b}f(x)dx
		\end{gathered}$
		\item $\begin{gathered}
			\int_{a}^{b}\left(f(x)+g(x)\right)dx=\int_{a}^{b}f(x)dx+\int_{a}^{b}g(x)dx
		\end{gathered}$
		\item $\begin{gathered}
			\int_{a}^{b}f(x)dx=\int_{a}^{c}f(x)dx+\int_{c}^{b}f(x)dx, c \in \left(a;b\right)
		\end{gathered}$\\
		\begin{tikzpicture}
			\draw[step=0.5,gray,very thin] (-0.5,-0.5) grid (5.5,4.5);
			\draw[->] (-0.5,0)--(4,0) node[right]{$x$};
			\draw[->] (0,-0.5)--(0,3) node[above]{$y$};
			\draw[line width=1pt] plot[domain=-0.2:2.55] (\x+0.7,{1+(\x-2)*(\x-2)*\x*\x});
			
%			\foreach \x in {-0.5,-0.1,...,1.5}{
%				\draw [dashed] (1+\x, 0)--(1+\x, {1+(\x-2)*(\x-2)*\x*\x});
%			};
			\foreach \x in {0,0.2,...,1}{
				\draw (0.5,\x) -- (\x+0.5, 0);
			};
			\foreach \x in {0,0.1,...,1.3}{
				\draw (\x+0.7, {1+(\x-2)*(\x-2)*\x*\x}) -- ({1+(\x-2)*(\x-2)*\x*\x},\x+0.7);
			};
%			\foreach \x in {-0.2,0,...,1.3}{
%				\draw (0.5, {1+(\x-2)*(\x-2)*\x*\x}) -- ({1+(\x-2)*(\x-2)*\x*\x},\x+0.7);
%			};
%			\foreach \x in {2.5,2.65,...,3.5}{
%				\draw ({\x}, -0.1) --(\x, 0.1);
%			};
			\draw (2, 0) node[anchor=north]{$c$} --(2, 1.85);
			\draw (0.5, 0) node[anchor=north]{$a$} --(0.5, 1.21);
			\draw (3.25, 0) node[anchor=north]{$b$} --(3.25, 2.9);
			
%			\draw (3.5, -0.1) node[anchor=north]{$x_{n}$} --(3.5, 0.1);
%			\draw [dashed] (3.5, 0)--(3.5, 2.56);
		\end{tikzpicture}
	\end{enumerate}
\end{document}
