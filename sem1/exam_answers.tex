\documentclass[12pt, fleqn]{article}
% Эта строка — комментарий, она не будет показана в выходном файле
\usepackage{cmap}
\usepackage{ucs}
\usepackage[utf8x]{inputenc} % Включаем поддержку UTF8
\usepackage[english, russian]{babel}  % Включаем пакет для поддержки русского языка
\usepackage{amsmath}
\usepackage{amssymb}
\usepackage{amsthm}
\usepackage{amsfonts}
\usepackage{array}
\usepackage[makeroom]{cancel}
\usepackage{enumerate}

\usepackage{tikz}
\title{Ответы МатАн}
\date{26.12.2024}
\author{Черепанов Илья}


\begin{document}
	\maketitle
	\clearpage
	\tableofcontents{}
	\clearpage
	\section{Множества на числовой прямой; определение окрестности конечной точки. Бесконечные точки.}
	\subsection{Определение множества.}
	\textbf{\textit{Множество}} -- совокупность объектов, объединенных по какому-то признаку.
	\\
	Объекты, из которых состоит множество, нахываются его \textbf{\textit{элементами}}.
	\\
	Множества принято обозначать заглавными буквами латинского алфавита $\{A,B,...,X,Y,...\}$, а их элементы -- малыми буквами $\{a,b,...,x,y,...\}$.
	\\
	Множество, не содержащее ни одного элемента, называется \textbf{\textit{пустым}}, обозначается символом $\varnothing$.
	\\
	Множество $A$ называется \textbf{\textit{подмножеством}} множества $B$, если каждый элемент множества $A$ является элементом множества $B$. Обозначается $A \subset B$.
	\\
	Говорят, что множества $A$ и $B$ \textbf{\textit{равны}} или \textbf{\textit{совападают}}, и пишут $A=B$, если $A \subset B$ и $B \subset A$. То есть, если множества состоят из одних и тех же элементов.
	\\
	\textbf{\textit{Объединением}} (или суммой) множеств A и B называется множество, состоящие из элементов, принадлежащих хотя бы одному их этих множеств. Обозначается $A \cup B$ (или$A+B$). Кратко можно записать $A\cup B = \{x\colon x\in A$ или $x \in B\}$
	\\
	\textbf{\textit{Пересечением}} (или произведением) множеств $A$ и $B$ называется множество, состоящее из элементов, каждый из которых принадлежит множеству $A$ и множеству $B$. Обозначают $A \cap B$ (или$A \cdot B$). Кратко можно записать $A\cap B = \{x\colon x\in A$ и $x \in B\}$
	\subsection{Числовые множества. Множества на прямой.}
	Множества, элементами которых явлвяются числа, называются \textbf{\textit{числовыми}}.\\
	Примеры числовых множеств: \\
	$\mathbb{N} = \{1; 2; 3; ...; n; ...\}$ - множество натуральных чисел.\\
	$\mathbb{Z} = \{\pm1; \pm2; \pm3; ...; \pm n; ...\}$ - множество целых чисел.\\
	$\mathbb{Q} = \{\frac m n\colon  m \in \mathbb{Z}, n \in \mathbb{N} \}$ - множество рациональных чисел.\\
	$\mathbb{R}$ -- множество вещественных чисел.\\
	Между этими множествами существует соотношение: \\
	$\mathbb{N \in Z \in Q \in R}$\\
	Действительные числа, не являющиеся рациональными, нызываются \textit{иррациональными}.\\\\
	Свойства $\mathbb{R}$:
	\begin{enumerate}
		\item Множество \textit{упорядоченное}:  для любых двух различных чисел a и b справедливо $a < b$ или $a > b$
		\item Множество \textit{плотное}:  между двумя раличными числами $a$ и $b$ содержится бесконечное множество действительных чисел.
		\item Множество \textit{непрерывное}. Пусть множество $\mathbb{R}$ разбито на два непустых класса $A$ и $B$ таких, что каждое действительное число содержится только в одном классе и для каждой пары чисел $a \in A$ и $b \in B$ выполнено неравенство $a<b$. Тогда существует единственное число $c$, удовлетворяющее неравенству $a \leq c \leq b  (\forall a \in A, \forall b \in B)$. Оно отделяет числа класса $A$ от чисел класса $B$. Число $c$ является либо наибольшим числом в классе $A$ (тогда в классе $B$ нет наименьшего числа), либо наименьшим числом в классе $B$ (тогда в класе $A$ нет наибольшего).
	\end{enumerate}
	Свойство непрерывности позволяет установить взаимно-однозначное соответствие между множеством всех действительных чисел и множеством всех точек прямой. Это означает, что каждому числу $x \in \mathbb{R}$ соответствует единственная точка числовой оси и наоборот.\\\\
	Пусть $a, b \in \mathbb{R}, a < b$\\
	\textit{Числовыми промежутками} (интервалами) называют подмножества всех действительных чисел, имеющих следующий вид: 
	\begin{description}
		\item $[a;b]=\{x\colon  a \leq x \leq b\}$ -- отрезок;
		\item $(a;b)=\{x\colon  a < x < b\}$ -- интервал;
		\item $[a;b)=\{x\colon a \leq x < b\}$;
		\item $(a;b]=\{x\colon a < x \leq b\}$ -- полуоткрытые интервалы (или полуоткрытые отрезки);
		\item $(-\infty;b]=\{x\colon x\leq b\}$;
		\item $(-\infty;b)=\{x\colon x<b\}$;
		\item $[a;\infty)=\{x\colon  x\geq a\}$;
		\item $(a;\infty)=\{x\colon  x> a\}$;
		\item $(-\infty;\infty)=\{x\colon  -\infty < x < \infty\} = \mathbb{R}$ -- бесконечные интервалы (промежутки);
	\end{description}
	\subsection{Окрестность конечной точки.}
	Пусть $x_0$ -- любое действительное число. \textbf{\textit{Окрестностью}} точки $x_0$ называется любой интервал $(a;b)$, содержащий точку $x_0$. В частоности итервал $(x_0-\varepsilon, x_0+\varepsilon)$, где $\varepsilon > 0$, называется \textbf{\textit{$\varepsilon$-окрестностью}} точки $x_0$. Число $x_0$ называется \textbf{\textit{центром}}, а число $\varepsilon$ -- \textit{радиусом}.
	$$\begin{tikzpicture}
		\draw (3,0) .. controls (4,0.4) .. (5,0);
		\draw (4,0.4)node[above]{$\varepsilon$};
		\draw (7,0) .. controls (6,0.4) .. (5,0);
		\draw (6,0.4)node[above]{$\varepsilon$};
		\draw (5.7,0.1) -- (5.7,-0.1) node[below=-0.9]{$x$}
		 (3,0.1)--(3,-0.1) node[below=-0.9]{$x_0-\varepsilon$}
		 (7,0.1)--(7,-0.1) node[below=-0.9]{$x_0+\varepsilon$}
		 [->] (0,0) -- (10,0) node[below=1] {$x$} coordinate(x axis);
		\fill (1, 0) circle (2pt) node[above=1] {$O$}
		 (5,0) circle (2pt) node [below=1]{$x_0$};		
	\end{tikzpicture}$$
	Если $x \in (x_0-\varepsilon;x_0+\varepsilon)$, то выполняется неравенство $x_0 - \varepsilon<x<x_o+\varepsilon$, или, что то же, $|x-x_0|<\varepsilon$. Выполнение последнего неравенства означает попадание точки $x$ в $\varepsilon$-окрестность точки $x_0$.
	\section{Определение функции. Область определения. Монотонная функция. Четная и  нечетная функции. Обратная функция. Сложная функция. Элементарные функции.
	}
	\subsection{Определение функции. Область определения}
	Пусть даны два непустых множества $X$ и $Y$. Соответствие $f$, которое каждому элементу $x \in X$ сопостовляет один и только один элемент $y \in Y$, называется \textbf{\textit{функцией}} и записывается $y=f(x), x \in X$ или $f\colon  X \rightarrow Y$. Говорят ещё, что функция $f$ \textbf{\textit{отображает}} множество $X$ на множество $Y$.\\
	Множество $X$ называется \textit{\textbf{областью определения}} функции $f$ и обозначается $D(f)$. Множество $Y$ называется \textit{множеством значений} функции $f$ и обозначается $E(f)$.
	\subsection{Четность/нечетность}
	Функция, определенная на множестве $D$, называется \textbf{\textit{четной}}, если $\forall x \in D$ выполняются условия $-x \in D$ и $f(-x)=f(x)$; \textbf{\textit{нечетной}}, если $\forall x \in D$ выполняются условия $-x \in D$ и $f(-x)=-f(x)$; остальные относятся к функциям \textbf{общего вида}.
	\subsection{Монотонность.}
	Пусть функция $y=f(x)$ определена на множестве $D$ и пусть $D_1\subset D$. Если для любых значений $x_1, x_2 \in D_1$ аргументов их неравенства $x_1 < x_2 $ вытекает неравенство
	\begin{description}
		\item $f(x_1) < f(x_2)$, то функция называется \textbf{\textit{возрастающей}} на множестве $D_1$; 
		\item $f(x_1)\leq f(x_2)$ -- \textbf{\textit{неубывающей}} на множестве $D_1$;
		\item $f(x_1)>f(x_2)$ -- \textbf{\textit{убывающей}} на множестве $D_1$;
		\item $f(x_1) \geq f(x_2)$ -- \textbf{\textit{невозрастающей}} на множестве $D_1$;
	\end{description}
	Возрастающие, невозрастающие, убывающие и неубывающие функции на множестве $D_1$ называются \textbf{\textit{монотонными}} на этом множестве, а возрастающие и убывающие - \textbf{\textit{строго монотонными}}. Интервалы, в которых функция монотонна, называются \textbf{\textit{интервалами монотонности}}
	\subsection{Обратная функция.}
	Пусть задана функция $y=f(x)$ с обласьтю определения $D$ и множеством значений $E$. Если каждому знчению $y\in E$ соответствует единственное значение $x\in D$, то определена функция $x = \varphi(y)$ с областью определения $E$ и множеством значений $D$. Такая функция $\varphi(y)$ называется \textbf{\textit{обратной}} к функции $f(x)$ и записывается как $x=\varphi(y)=f^{-1}(y)$. Про функции $y=f(x)$ и $x=\varphi(y)$ говорят, что они являются взаимно обратными. Чтобы найти функцию $x=\varphi(y)$ достаточно решить уравнение $y=f(x)$ относительно $x$ (если это возможно).\\\\
	Из определения обратной функции вытекает, что функция $y=f(x)$ имеет обратную тогда и только тогда, когда функция $f(x)$ задает взаимно однозначное соответствие между множествами $D$ и $E$. Отсюда следует, что любая \textbf{\textit{строго монотонная функция имеет обратную}}.
	\subsection{Сложная функция.}
	Пусть функция $y=f(u)$ определена на множестве $D$, а функция $u=\varphi(x)$ на множестве $D_1$, причем для $\forall x \in D_1$ соответствующее значение $u=\varphi(x) \in D$. Тогда на множестве $D_1$ определена функция $y=f(\varphi(x))$, которая называется \textbf{\textit{сложной функцией}} от $(x)$ (или \textbf{\textit{суперпозицией}} заданных функций).\\
	Переменную $u=\varphi(x)$ называют \textit{промежуточным аргументом} сложной функции.
	\subsection{Элементарные функции.}
	Основные элементарыне функции: 
	\begin{enumerate}
		\item \textit{Показательная} функция $y=a^x, a>0, a\neq0$
		\item \textit{Степенная} функция $y=x^\alpha, \alpha \in \mathbb{R}$
		\item \textit{Логарифмическая} функция $y=\log_{a}x, a > 0, a \neq 1$
		\item \textit{Тригонометрические }функции $y=\sin x, y=\cos x, y=\tg x, y =\ctg x$
		\item \textit{Обратные тригонометрические }функции $y=\arcsin x, y=\arccos x, y=\arctg x, y =\arcctg x$
	\end{enumerate}
	Функция, задаваемая одной формулой, составленной из основных элементарных функций и постоянных с помощью конечного числа арифметических операций (сложения, вычитания, умножения деления) и операций взятия функции от функции, называется \textbf{\textit{элементарной функцией}}.
	\section{Определение предела последовательности; определение предела функции в точке (конечной и бесконечной). Простейшие пределы $\lim_{x\to a} c=c$ и $\lim_{x \to a} x=a$ . Единственность конечного предела.}
	\subsection{Предел последовательности}
	Под \textbf{\textit{числовой последовательностью}} $x_1, x_2, x_3,...,x_n,...$ понимается функция $x_n = f(n)$, заданная на множестве $\mathbb{N}$, кратко обозначается $\{x_n\}$.\\\\
	Число $a$ называется \textbf{\textit{пределом последовательности}} $\{x_n\}$, если для любого положительного числа $\varepsilon$ найдется такое натуральное число $N$, что при всех $n>N$ выполняется неравенство $|x_n-a|<\varepsilon$. 
	В этом случае пишут $$\lim_{x\to \infty} x_n=\lim x_n=a$$ или $x_n \to a$.\\
	Короткое определение предела:
	$$\left( \forall \varepsilon > 0  \exists N \colon \forall n > N \Longrightarrow \left|x_n-a \right| < \varepsilon \right) \Longleftrightarrow \lim_{n\to\infty}x_n=a $$
	Геометрическая интерпритация:
	$$\begin{tikzpicture}
		\draw 
		(3,0.2)--(3,-0.2) node[below=-2]{$a-\varepsilon$}
		(7,0.2)--(7,-0.2) node[below=-2]{$a+\varepsilon$}
		\foreach \x in {70,68.5 , ..., 10.5}{
				(5+200/\x^2,0.1)--(5+200/\x^2,-0.1) 
				(5-200/\x^2,0.1)--(5-200/\x^2,-0.1) 
			}
		[->] (0,0) -- (10,0) node[below=5] {$x$} coordinate(x axis);
		\fill (1, 0) circle (2pt) node[above=1] {$O$}
		(5,0) circle (2pt) node [below=5]{$a$};
	\end{tikzpicture}$$
	Чило $a$ называется пределом последовательности $\{x_n\}$, если для любой $\varepsilon$-окрестности точки $a$ найдется $N \in \mathbb{N}$, что все значения $x_n$, для которых $n > N$, попадут в $\varepsilon$-окрестность точки $a$.
	\subsection{Предел функции в конечной точке}
	\paragraph{Определение 1 (на языке последовательностей или по Гейне).} Число $A$ называется \textbf{\textit{пределом функции}} $y=f(x)$ в точке $x_0$ (или при $x \to x_0$), если для любой последовательности допустимых значений аргумента $x_n, n \in \mathbb{N} (x_n \neq x_0)$, сходящейся к $x_0$ (т.е. $\lim_{n \to \infty}x_n=x_0$), последовательность соотоветвующих значений функции $f(x_n), n \in \mathbb{N}$, сходится к числу $A$ (т.е. $\lim_{n \to \infty}f(x_n)=A$).\\
	В этом случае пишут $\lim_{x\to x_0}f(x)=A$ или $f(x) \to A $ при $x \to x_0$. Геометрический смысл предела функции $\lim_{x \to x_0}f(x) = A$ означает, что для всех точек $x$, достаточно близких к точке $x_0$, соответствующие значения функции как угодно мало отличаются от числа $A$.
	\paragraph{Определение 2 (на языке $\varepsilon$-$\delta$ или по Коши)}
	Число $A$ наызается пределом функции в точке $x_0$(или при $x \to x_0$), если для любого положительного $\varepsilon$ найдется такое положительное число $\delta$, что для всех $x \neq x_0$, удовлетворяющих неравенству $|x-x_0|<\delta$, выполняется неравенство $|f(x) - A| < \varepsilon$.
	Записывают $\lim_{x \to x_0} f(x) = A$. Это определение коротко можно записать так $$\left( \forall \varepsilon > 0 \exists \delta>0 \forall x \colon|x-x_0| < \delta, x \neq x_0 \Longrightarrow |f(x) - A| < \varepsilon \right) \Longleftrightarrow\lim_{x\to x_0}f(x) = A$$
	Геометрический смысл предела функции
	 $A=\lim_{x \to x_0}f(x)$, если для любой $\varepsilon$-окрестности точки $A$ найдется такая $\delta$-окрестность точки $x_0$, что для всех $x\neq x_0$ из этой $\delta$-окрестности соответствующие значения функции $f(x)$ лежат в $\varepsilon$- окрестности точки $A$. Иными словами, точки графика функции $y=f(x)$ лежат внутри полосы шириной $2\varepsilon$, ограниченной прямыми $y=A+\varepsilon, y=A-\varepsilon$. Очевидно, что величина $\delta$ зависит от выбора $\varepsilon$, поэтому пишут $\delta = \delta(\varepsilon).$
	\subsection{Предел функции в бесконечной точке}
	Пусть функция $y=f(x)$ определена в промежутке $(-\infty;\infty)$. Число $A$ называется \textbf{\textit{Пределом функции $f(x)$ при $x \to \infty$}}, если для любого положительного числа $\varepsilon$ существует такое счисло $M=M(\varepsilon) > 0$, что при всех $x$, удовлетворяющих неравенству $|x| > M$ выполняется неравенство $|f(x) - A| < \varepsilon$.\\ Коротко:
	$$\left( \forall\varepsilon>0\exists M>0 \forall x\colon |x|>M \Longrightarrow |f(x)-A| < \varepsilon \right) \Longleftrightarrow \lim_{x\to\infty}f(x)=A$$ 
	Если $x\to +\infty$, то пишут $A=\lim_{x\to+\infty}f(x)$, если $x \to -\infty$, то -- $A=\lim_{x\to-\infty}f(x).$ Геометрический смысл этого определения таков:
	для $\forall \varepsilon > 0 \exists M > 0$, что при $x\in \left(-\infty; -M\right)$ или $x \in \left(M;+\infty\right)$ соответствующие значения функции $f(x)$ попадают в $\varepsilon$-окрестность точки $A$, т.е. точки графика лежат в полосе шириной $2\varepsilon$, ограниченной прямыми $y=A+\varepsilon$ и $y=A-\varepsilon$.
	\subsection{Простейшие пределы}
	\begin{enumerate}
		\item $\lim_{x\to a}x=a$.\\
		\textit{Доказательство}:
		$$\forall \varepsilon > 0 \exists \delta > 0 \colon x \in \mathring{U}_\delta(a) \curvearrowright f(x)=x\in \mathring{U}_\varepsilon(A)$$ $$ \delta = \varepsilon$$
		$\blacksquare$
		\item $f(x) = c=const \Longrightarrow \lim_{x\to a} c=c$\\
		\textit{Доказательство}:
		$$\forall \varepsilon > 0 \exists \delta > 0 \colon x \in \mathring{U}_\delta (a) \curvearrowright f(x)=c \in \mathring{U}_\varepsilon(c)$$ 
		$\delta$ -- любое, чтобы окрестность входила в область опредления.
		$\blacksquare$
	\end{enumerate}
	\subsection{Единственность конечного предела.}
	$\lim_{x\to a}f(x) = A $-- единственный предел.\\
	\textit{Доказательство}:\\
	Допустим $f(x) \to A; x \to a$\\
	$f(x)\to B; x \to a$\\
	$B>A$\\
	\begin{equation}
		\forall \varepsilon > 0 \exists \delta_1\colon x\in \mathring{U}_{\delta_1}(a) \curvearrowright \left( f(x) \in \mathring{U}_\varepsilon(A) \right) \Leftrightarrow \left( A - \varepsilon < f(x) < A+ \varepsilon \right) \label{first}
	\end{equation} 
	\begin{equation}
		\forall \varepsilon > 0 \exists \delta_1\colon x\in \mathring{U}_{\delta_2}(a) \curvearrowright \left( f(x) \in \mathring{U}_\varepsilon(B) \right) \Leftrightarrow \left( B - \varepsilon < f(x) < B+ \varepsilon \right) \label{second}
	\end{equation} 
	$\delta = \min\{\delta_1, \delta_2\} \Rightarrow$ при $x\in \mathring{U}_\delta$ выполнены правила $(\ref{first})$ и $(\ref{second})$\\
	Возьмем $\varepsilon = \frac{B-A}{2}$\\
	\begin{align*}
		&\left.
			\begin{gathered}
				\begin{aligned}
					&(\ref{first}) \colon \dots < f(x)<A+\frac{B-A}{2}=\frac{B+A}{2}\\
					&(\ref{second}) \colon B-\frac{B-A}{2} < f(x)<A+\dots \\
				\end{aligned}
			\end{gathered}
		\right\} \Rightarrow \frac{B+A}{2}< f(x)< \frac{B+A}{2}\Rightarrow \text{Предел единственный}\\
		&\blacksquare\\
	\end{align*}
\section{Свойства функций, стремящихся к конечному пределу (ограниченность функции, имеющей конечный предел; теорема о сжатой переменной; предельный переход в неравенстве).}
\begin{enumerate}
	\item $\lim_{x \to a} f(x)=A$ -- единственный
	\item $\lim_{x \to a} f(x) =A$ -- конечный $\Rightarrow \exists \mathring{U}(a)\colon f(x)$ -- ограничена в $\mathring{U}(a)$
	\item Предельный переход в неравенстве
	\begin{multline*}
		\lim_{x \to a}f(x)=A, \lim_{x \to a}g(x)=B\\
		f(x)\leq g(x), x \in \mathring{U}(a) \Rightarrow \lim_{x \to a}f(x)\leq \lim_{x \to a}g(x)\\
		A \leq B\\
	\end{multline*}
	\textit{Доказательство:}
	\begin{multline*}
			\forall \varepsilon > 0 \exists \delta_1>0\colon x\in \mathring{U}_{\delta_1}(a) \curvearrowright \left( A - \varepsilon < f(x) < A+ \varepsilon \right)\\
			\forall \varepsilon > 0 \exists \delta_1>0\colon x\in \mathring{U}_{\delta_2}(a) \curvearrowright \left( B - \varepsilon < f(x) < B+ \varepsilon \right)\\
			\delta = min(\delta_1, \delta_2)\\
			A-\varepsilon<f(x\leq g(x)<B+\varepsilon)\\
			A<B+2\varepsilon\\
			A\leq B\\
			\blacksquare\\
	\end{multline*}
	\item $f(x)\leq\varphi(x)\leq g(x)$ -- Т. о сжатой переменной (двух милиционеров)
	\begin{multline*}
		\left.
			\begin{gathered}
				\lim_{x \to a}f(x)=A\\
				\lim_{x \to a}g(x)=A\\
			\end{gathered}
		\right\}\Rightarrow \lim_{x \to a}\varphi(x) = A\\
	\end{multline*}
	\textit{Доказательство:}
	\begin{multline*}
		\forall \varepsilon >0 \exists \delta>0: x \in \mathring{U}_\delta(a)\colon\begin{gathered}
			A-\varepsilon < f(x)<A+\varepsilon\\
			A-\varepsilon < g(x)<A+\varepsilon\\
		\end{gathered}\\
		A-\varepsilon < f(x)\leq\varphi(x)\leq g(x)<A+\varepsilon\\
		\blacksquare\\
	\end{multline*}
\end{enumerate}
\section{Односторонние пределы функции в точке.  Необходимое и достаточное условие существования конечного предела(использующее односторонние пределы). Первый замечательный предел: $\lim_{x\to 0}$$\frac{\sin x}{x}$.} 
\subsection{Односторонние пределы}
$f(x)$ -- определена на $(b;a), a$ -- конечная точка.\\
Число $A$ -- предел $f(x)$ слева в точке $a$, если
$$\forall \varepsilon >0 \exists\delta>0\colon x\in(a-\delta;a)\curvearrowright f(x)\in U_\varepsilon(A)$$
Обозначение:
\begin{multline*}
	\lim_{x \to a-}f(x)\\
	\lim_{x\to a-0}f(x)\\
\end{multline*}
Справа аналогично\\
\begin{multline*}
	\lim_{x \to a+}f(x)\\
	\lim_{x\to a+0}f(x)\\
\end{multline*}
\begin{tikzpicture}
	\draw[->] (-1,0)--(5,0) node[right]{$x$};
	\draw[->] (0,-1)--(0,5) node[above]{$y$};
	%\draw plot[domain=-1:-0.5] (\x,{1.5+\x/(\x+1.1)});
	\draw[line width=1pt] plot[domain=-0.5:2] (\x,{1+\x*\x/(\x+2)});
	\draw[line width=1pt] plot[domain=2:4] (\x, {(4-(\x-2.5)*(\x-2.5))});
	\draw [dashed] (0,2)--(2,2);
	\draw [dashed] (0,3.75)--(2,3.75);
	\draw [dashed] (2,0)--(2,3.75);
	%\draw [dashed, blue] (2,0)--(2,1);
	%\draw [dashed, red] (0,3)--(1,3);
	%\draw [dashed, red] (1,0)--(1,3);
	\draw [line width=2pt](2,0.1)--(2,-0.1) node [anchor=north]{\textbf{a}};
	%\draw [line width=2pt] (1,0.1)--(1,-0.1) node [anchor=north]{\textbf{c}};
	\draw [line width=2pt] (0.1,2)--(-0.1,2) node [anchor=east]{$f(x-0)$};
		\draw [line width=2pt] (0.1,3.75)--(-0.1, 3.75) node [anchor=east]{$f(x+0)$};
	%\draw [line width=2pt] (0.1,3)--(-0.1,3) node [anchor=east]{\textbf{d}};
	
	%\draw[line width=2pt,blue,-stealth](0,0)--(2,1) node[anchor=south west]{$\boldsymbol{\alpha}$};
	%\draw[line width=2pt,red,-stealth](0,0)--(1, 3) node[anchor=north east]{$\boldsymbol{\beta}$};
\end{tikzpicture}\\
\subsection{Необходимое и достаточное условие существования конечного предела}
$$\lim_{x \to a}f(x)=A \Leftrightarrow \lim_{x \to a-}f(x)=\lim_{x \to a+}f(x)=A$$
\subsection{Первый замечательный предел}
\begin{multline*}
	\lim_{x \to 0}\frac{\sin x}{x}=1\\
	\lim_{x \to 0}\frac{\tg x}{x}=1\\
	\lim_{x \to 0}\frac{1 - \cos x}{x^2}=\frac{1}{2}\\
	\lim_{x \to 0}\frac{\arcsin x}{x}=1\\
	\lim_{x \to 0}\frac{\arctg x}{x}=1\\
\end{multline*}
\section{Теорема о монотонной ограниченной функции и последовательности (без доказательства). Второй  замечательный предел. Доказательство существования .  }
\subsection{Теорема о монотонной ограниченной последовательности}
Всякая монотонная ограниченная последовательность имеет предел.\\
\subsection{Теорема о монотонной ограниченной функции}
Если функция $f(x)$ монотонна и ограничена при $x<x_0$ или $x>x_0$, то существует соответственно ее левый предел $\lim_{x \to x_0-0}f(x)=f(x_0-0)$ или ее правый предел $\lim_{x \to x_0+0}f(x)=f(x_0+0)$
\subsection{Второй замечательный предел}
$$\lim_{x\to\infty}\left(1+\frac{1}{x}\right)^x=e$$
\textit{Доказательство существования:}
\begin{multline*}
	\left(1+\frac{1}{n}\right)^n=\sum_{k=0}^{n}C_n^k \frac{1}{n^k}\cdot 1^{n-k}=\sum_{k=0}^{n}\frac{n!}{\left(n-k\right)!\cdot k!}\cdot \frac{1}{n^k}=\\
	=\sum_{k=0}^{n}\frac{(n-k+1)(n-k+2)\dots(n-(n-1))\cdot n}{k!n^{k}}=\\
	=1+\sum_{k=1}^{n}\frac{(n-k+1)(n-k+2)\dots(n-(n-1))}{k!n^{k-1}}=\\
	=1+1+\frac{n-1}{2n}+\frac{(n-1)(n-2)}{3!n^2}+\dots+\frac{(n-(n-1))(n-(n-2)\dots(n-2)(n-1)}{n!n^{n-1}}\\
	\blacksquare\\
\end{multline*}
\textit{Следствия:}
\begin{enumerate}
	\item $\lim_{x\to\infty}\left(1+\frac{1}{x}\right)^x=e$
	\item $\lim_{x\to 0}\left(1+x\right)^{\frac{1}{x}}=e$
\end{enumerate}
\section{Бесконечно малые функции, их свойства. Необходимое и достаточное условие стремления функции к конечному пределу.}
\subsection{Б.М.Ф.}
$$\lim_{x\to x_0}f(x)=0$$
\textit{Свойства б.м.ф.}
\begin{enumerate}
	\item $f(x)$ -- б. м. в точке $a \Rightarrow c\cdot f(x)$ -- б. м. в точке $a$
	\item $f(x)$ -- б. м. в точке $a; g(x)$ -- ограничена в $U(a)$ или $g(x)$ -- б.м. в точке $a \Rightarrow f(x)\cdot g(x)$ -- б.м. в точке $a$
	\item $f(x)$ -- б. м. в точке $a; g(x)$ -- б.м. в точке $a \Rightarrow \lim_{x\to a}\left(f(x)+g(x)\right)=0$
\end{enumerate}
\subsection{Необходимое и достаточное условие стремления функции к конечному пределу}
$$\lim_{x \to a}A_{\text{кон}} \Leftrightarrow \lim_{x\to a-0} f(x)=\lim_{x\to a+0}f(x)$$
\section{Бесконечно большие функции их свойства. Теорема о связи бесконечно малой и бесконечно большой функций.}
\subsection{Б.Б.Ф.}
$$\lim_{x\to a}f(x)=\infty$$
\textit{Свойства Б.Б.Ф.}
\begin{enumerate}
	\item $cf(x)$ -- б.б. в точке $а$ $(c\neq 0)$
	\item $f(x) + A_{\text{кон}}$ -- б.б. в точке $a$
	\item $f(x)\cdot g(x)$ --б.б. в точке $a$ ($g(x)$ -- б.б.)
	\item \begin{multline*}
		\left. \begin{gathered}
			\lim_{x\to a}f(x)=+\infty\\
			\lim_{x\to a}g(x)=+\infty\\
		\end{gathered}\right\}\lim_{x\to a}\left(f(x)+g(x)\right)=+\infty\\
	\end{multline*}
\end{enumerate}
\subsection{Теорема о связи б.м. и б.б. функций}
\begin{itemize}
	\item $f(x)$ -- б.б. в точке $a \Rightarrow \frac{1}{f(x)}$ -- б.м. в точке $a$
	\item $f(x)$ -- б.м. в точке $a \Rightarrow \frac{1}{f(x)}$ -- б.б. в точке $a$
\end{itemize}
\section{Предел суммы, произведения и частного функций, стремящихся к конечным пределам.}
\begin{align*}
	\text{Пусть  }&\lim_{x\to a} f(x) = A &&\lim_{x\to a} g(x)= B&&
\end{align*}
	\begin{enumerate}
		\item $\begin{gathered}\lim_{x\to a} cf(x)=cA\end{gathered}$
		\item $\begin{gathered}\lim_{x\to a} \left(f(x) + g(x)\right)=A+B\end{gathered}$
		\item $\begin{gathered}\lim_{x\to a} \left(f(x) \cdot g(x)\right)=AB\end{gathered}$
		\item $\begin{gathered}\lim_{x\to a} \left(\frac{f(x)}{g(x)}\right)=\frac{A}{B}\end{gathered}$
\end{enumerate}
\section{Сравнение бесконечно малых. Эквивалентные бесконечно малые. Таблица эквивалентных бесконечно малых(доказательство). Теорема о замене бесконечно малой на эквивалентную при вычислении пределов.}
\subsection{Сравнение б.м.ф.}
$f(x), g(x)$ -- б.м. в точке $a$
\begin{align*}
	\lim_{x\to a}\frac{f(x)}{g(x)} = \begin{cases}
		\begin{aligned}
		&0,&& f(x)\text{ более высокого порядка}\\
		&A\neq0,&&f(x) \text{ и } g(x) \text{ одного порядка}\\
		&\infty,&&g(x) \text{ более высокого порядка}\\
		&\nexists,&& \text{несравнимы}\\
		\end{aligned}
	\end{cases}
\end{align*}
\subsection{Эквивалентные б.м.ф.}
Предел отношения б.м.ф. ге изменится, если каджую из них (или одну) заменить на эквивалентную б.м.ф.\\
Если $\begin{gathered}
	\lim_{x \to a}\frac{\alpha(x)}{\beta(x)}=1
\end{gathered}$, то $\alpha$ и $\beta$ эквивалентны в точке $a$.\\
\textit{Доказательство:}
\begin{multline*}
	f(x) \sim \alpha(x)\\
	g(x) \sim \beta(x)\\
	\lim_{x\to a}\frac{f(x)}{g(x)}=\lim_{x\to a} \frac{f(x)}{g(x)}\cdot\frac{\alpha(x)}{\alpha(x)}\cdot \frac{\beta(x)}{\beta(x)}=\lim_{x\to a}\cancelto{1}{\frac{f(x)}{\alpha(x)}}\cdot\cancelto{1}{\frac{\beta(x)}{g(x)}}\cdot\frac{\alpha(x)}{\beta(x)}=\\
	=\lim_{x\to a}\frac{\alpha(x)}{\beta(x)}\\
\end{multline*}
\subsection{Таблица эквивалентных б.м.ф.}
\begin{enumerate}
	\item $\sin \alpha(x) \sim \alpha(x)$
	\item $\tg \alpha(x) \sim \alpha(x)$
	\item $1-\cos \alpha(x)\sim \frac{\left(\alpha(x)\right)^2}{2}$
	\item $\arcsin \alpha(x)\sim \alpha(x)$
	\item $\arctg \alpha(x)\sim\alpha(x)$
	\item $e^{\alpha(x)}-1\sim \alpha(x)$
	\item $\ln(1+\alpha(x))\sim \alpha(x)$
	\item $ \left(1+\alpha(x)^n\right)-1\sim xn$
\end{enumerate}
\section{Непрерывность функции в точке. Необходимое и достаточное условия непрерывности функции в точке(использующие приращения).}
\subsection{Непрерывность функции в точке}
Функция $y=f(x)$ определена в $U(a)$\\
$f(x)$ -- непрерывна в точке $a$, если существует предел, и он равен значнию функции в этой точке $\begin{gathered}
	\lim_{x\to a} f(x)= f(a)
\end{gathered}$
\subsection{Необходимое и достаточное условия непрерывности функции в точке}
\begin{multline*}
	\varDelta f(x) = f(x_1) - f(x)\\
	x_1=x+\varDelta x\\
	\varDelta x = x_1-x\\
	\lim_{\varDelta x\to0}\varDelta f(x) = 0 \Leftrightarrow f(x) \text{ -- непрерывна в точке } x\\
\end{multline*}
\begin{tikzpicture}
	\draw[->] (-1,0)--(5,0) node[right]{$x$};
	\draw[->] (0,-1)--(0,5) node[above]{$y$};
	%\draw plot[domain=-1:-0.5] (\x,{1.5+\x/(\x+1.1)});
	\draw[line width=1pt] plot[domain=-0.8:4] (\x,{1+\x*\x/(\x+2)});
%	\draw[line width=1pt] plot[domain=2:4] (\x, {(4-(\x-2.5)*(\x-2.5))});
	\draw [dashed] (0,2)--(2,2);
	\draw [dashed] (0,2.8)--(3,2.8);
	\draw [dashed] (3,0)--(3,2.8);
	\draw [dashed] (2,0)--(2,2);
	%\draw [dashed, blue] (2,0)--(2,1);
	%\draw [dashed, red] (0,3)--(1,3);
	%\draw [dashed, red] (1,0)--(1,3);
	\draw [line width=2pt](2,0.1)--(2,-0.1) node [below=3]{$x$};
	\draw [line width=2pt](3,0.1)--(3,-0.1) node [below]{$x+\varDelta x$};
	%\draw [line width=2pt] (1,0.1)--(1,-0.1) node [anchor=north]{\textbf{c}};
	\draw [line width=2pt] (0.1,2)--(-0.1,2);
	\draw [line width=2pt] (0.1,2.8)--(-0.1, 2.8);
	%\draw [line width=2pt] (0.1,3)--(-0.1,3) node [anchor=east]{\textbf{d}};
	
	%\draw[line width=2pt,blue,-stealth](0,0)--(2,1) node[anchor=south west]{$\boldsymbol{\alpha}$};
	%\draw[line width=2pt,red,-stealth](0,0)--(1, 3) node[anchor=north east]{$\boldsymbol{\beta}$};
\end{tikzpicture}\\
\subsection{Свойства непрерывности функции в точке}
\begin{enumerate}
	\item $f(x)$ -- непрерывна в точке $x \Rightarrow cf(x)$ -- непрерывна в точке $x$
	\item $f(x), g(x)$ -- непрерывны в точке $x \Rightarrow \left. \begin{gathered}
		f(x)+ g(x)\\
		f(x)\cdot g(x)\\
	\end{gathered}\right\}$ -- непрерывны в точке $x$
	\item $f(x)$ -- непрерывна в точке $x_0, f(x_0)\neq 0 \Rightarrow \exists U(x_0)\colon f(x)$ -- того же знака, что $f(x_0)$\\
	\textit{Доказательство:}\\
	Пусть $f(x_0) > 0$:
	\begin{multline*}
		\lim_{x\to x_0} = f(x_0) \text{, т.е. } \forall \varepsilon > 0 \exists \delta>0\colon x\in U_\delta(x_0)\\
		f(x_0) - \varepsilon < f(x) < f(x_0)+\varepsilon\\
		\text{Выберем } \varepsilon\colon f(x_0)-\varepsilon > 0 \Rightarrow \delta \text{ -- искомое}\\
		\blacksquare\\
	\end{multline*}
	\item $f(x), g(x)$ -- непрерывны в точке $x_0$, $g(x)\neq 0\colon$\\
	$\frac{f(x)}{g(x)}$ -- непрерывна в точке $x_0$
	\item $u(x)$ -- непрерывна в точке $x_0, u(x_0)=u_0\colon$\\
	$f(u)$ -- непрерывна в точке $u_0 \Rightarrow f(x)=f(u(x))$ -- непрерывна в точке $x_0$
	\item Все элементарные функции непрерывны на своей области определения.
\end{enumerate}
\section{Свойства функций, непрерывных в точке. Свойства функций, непрерывных на замкнутом промежутке(без доказательства)}
\subsection{Свойства функций, непрерывных в точке}
\begin{enumerate}
	\item Если $f(x)$ -- непрерывна в точке $a$, то существует $U(a)$, в которой $f(x)$ ограничена.
	\item Если $f(x)$ -- непрерывна в точке $a, f(a)\neq 0$, то в некоторой $U(a) f(x)$ сохраняет свой знак.
\end{enumerate}
\subsection{Свойства функций, непрерывных на замкнутом промежутке}
$f(x)$ --непрерывна на промежутке $[a;b]$, если:
\begin{enumerate}[\space\space\space1)]
	\item $f(x)$ -- непрерывна на $(a;b)$
	\item $f(a+0)=f(a)$
	\item $f(b-0)=f(b)$
\end{enumerate}
\subsection*{1 теорема Вейерштрасса}
Если $f(x)$ непрерывна на $[a;b]$, то она на нем ограничена.
\subsection*{2 теорема Вейерштрасса}
Если $f(x)$ непрерывна на $[a;b]$, то она достигает на нем наибольшее и наименьшее свое значение.
\subsection*{1 теорема Больцано-Коши}
$f(x)$ непрерывна на $[a;b]$\\
$f(a)\cdot f(b)<0\Rightarrow\exists c\in[a;b]\colon f(c)=0$
\subsection*{3 теорема Больцано-Коши}
$f(x)$ непрерывна на $[a;b]$\\
$f(a)=A , f(b)=B \Rightarrow A\neq B \forall C \in (A;B)\exists c \in (a;b)\colon f(c)=C$
\section{Классификация точек разрыва функции.}
$f(x)$ определена в $U(a)$ или $\mathring{U}(a)$\\
$x=a$ -- точка разрыва функции $f(x)$, если выполняется хотя бы одно из условий:
\begin{enumerate}[\space1)]
	\item Функция не определена в точке $a$
	\item Функция определена в точке $a$, но не существует $\begin{gathered}
		\lim_{x\to a}f(x)
	\end{gathered}$
	\item Функция определена, существуют пределы $\begin{gathered}
		\lim_{x\to a-}f(x)
	\end{gathered}$ и $\begin{gathered}
	\lim_{x\to a+}f(x)
	\end{gathered}$, но не выполняется условие $\begin{gathered}
	\lim_{x\to a-}f(x)
	\end{gathered} = \begin{gathered}
	\lim_{x\to a+}f(x)
	\end{gathered} = f(a)$
\end{enumerate}
\subsection*{Классификация точек разрыва функции}
\begin{enumerate}[I.]
	\item $a$ -- точка разрыва $I$-го рода, если $\begin{gathered}
		\lim_{x\to a-0}f(x)
	\end{gathered}$ и $\begin{gathered}
	\lim_{x\to a+0}f(x)
	\end{gathered}$ -- конечные
	\begin{enumerate}[1)]
		\item Если $\begin{gathered}
			\lim_{x\to a-}f(x)
		\end{gathered}=\begin{gathered}
		\lim_{x\to a+}f(x)
		\end{gathered}$, то $a$ -- точка устранимого разрыва.
		\begin{tikzpicture}
			\draw[->] (-0.5,0)--(3,0) node[right]{$x$};
			\draw[->] (0,-0.5)--(0,3) node[above]{$y$};
			\draw[line width=1pt] plot[domain=-0.4:1.9] (\x,{1+\x/4});
			\draw[line width=1pt] plot[domain=2.1:3] (\x,{1+\x/4});

			\draw [dashed] (2,0)--(2,1.5);
			\draw (2,1.5) circle (3pt);
			\draw [line width=2pt](2,0.1)--(2,-0.1) node [below=3]{$a$};
		\end{tikzpicture}\\
				\item Если $\begin{gathered}
			\lim_{x\to a-}f(x)
		\end{gathered}\neq\begin{gathered}
			\lim_{x\to a+}f(x)
		\end{gathered}$, то $a$ -- точка конечного разрыва.
		\begin{tikzpicture}
			\draw[->] (-0.5,0)--(3,0) node[right]{$x$};
			\draw[->] (0,-0.5)--(0,3) node[above]{$y$};
			\draw[line width=1pt] plot[domain=-0.4:1.9] (\x,{1+\x/4});
			\draw[line width=1pt] plot[domain=2.06:3] (\x,{-1.5+\x});
			
			\draw [dashed] (2,0)--(2,1.5);
			\draw (2,1.5) circle (3pt);
			\draw (2,0.5) circle (3pt);
			\draw [line width=2pt](2,0.1)--(2,-0.1) node [below=3]{$a$};
		\end{tikzpicture}\\
	\end{enumerate}
	\item $a$ -- точка разрыва $II$-го рода, если хотя бы один из $\begin{gathered}
		\lim_{x\to a-}f(x)
	\end{gathered}$, $\begin{gathered}
	\lim_{x\to a+}f(x)
	\end{gathered}$ -- бесконечный или не существует\\
	\begin{tikzpicture}
		\draw[->] (-0.5,0)--(3,0) node[right]{$x$};
		\draw[->] (0,-0.5)--(0,3) node[above]{$y$};
		\draw[line width=1pt] plot[domain=-0.4:1.9] (\x,{1+(\x-2)*(\x-2)*\x*\x});
		\draw[line width=1pt] plot[domain=2.05:3] (\x,{0.8/(\x-1.8)});			
		\draw [dashed] (2,0)--(2,3);
		\draw (2,1) circle (3pt);
		\draw [line width=2pt](2,0.1)--(2,-0.1) node [below=3]{$a$};
	\end{tikzpicture}\\
\end{enumerate}
\section{Определение производной. Примеры нахождения производной с помощью определения.}
$f(x)$ -- определена в $U(x)$\\
$\varDelta x \colon x + \varDelta x \in U(x)$\\
$\varDelta f(x) = f(x-\varDelta x)-f(x)$\\
Предел отношения приращения функции к приращению аргумента называется \textbf{\textit{производной}}.\\
$$\lim_{\varDelta x \to 0}\frac{\varDelta f(x)}{\varDelta x}=\lim_{\varDelta x \to 0} \frac{f(x+\varDelta x)-f(x)}{\varDelta x} = f^\prime(x)=\frac{df(x)}{d(x)}$$
\begin{tikzpicture}
	\draw[->] (-1,0)--(5,0) node[right]{$x$};
	\draw[->] (0,-1)--(0,5) node[above]{$y$};

	\draw[line width=1pt] plot[domain=-0.8:4] (\x,{1+\x*\x/(\x+2)});

	\draw [dashed] (0,2)--(2,2);
	\draw [dashed] (0,2.8)--(3,2.8);
	\draw [dashed] (3,0)--(3,2.8);
	\draw [dashed] (2,0)--(2,2);
	
	\draw [line width=2pt](2,0.1)--(2,-0.1) node [below=3]{$x$};
	\draw [line width=2pt](3,0.1)--(3,-0.1) node [below]{$x+\varDelta x$};
	\draw [line width=2pt] (0.1,2)--(-0.1,2) node [left] {$f(x)$};
	\draw [line width=2pt] (0.1,2.8)--(-0.1, 2.8) node [left] {$f(x+\varDelta x)$};
\end{tikzpicture}
\section{Геометрический и механический смысл производной. Уравнение касательной к графику функции.}
\subsection{Геометрический смысл производной}
\begin{tikzpicture}
%	\draw[step=1.0,gray! 40,thin] (-5,-1) grid (5.5,4.5);
	\draw[->] (-5,0)--(5,0) node[right]{$x$};
	\draw[->] (0,-1)--(0,5) node[above]{$y$};
	
	\draw[line width=1pt] plot[domain=-1:5] (\x, {1+(\x - 1)*\x/(\x+4)});
	\draw[] plot[domain=-5:5] (\x, {(\x+4)/5});
	\draw[] plot[domain=-2:5] (\x, {(0.5*\x+0.5)});
	\draw (-0.1,0) arc (0:24:1);
	\draw (0.3, 0)node[above=-1]{$\alpha$};
	\draw (-2,0) arc (0:23.2:1);
	\draw (-1.5, 0)node[above=-1]{$\varphi$};
%	
	\draw [dashed] (0,1)--(4,1)[right] node{A};
	\draw [dashed] (0,2.5)--(4,2.5);
	\draw [dashed] (4,0)--(4,2.5) node [above] {$M$};
	\draw [dashed] (1,0)--(1,1) node[above]{$M_0$};
	

	\draw [line width=2pt](1,0.1)--(1,-0.1) node [below=3]{$x$};
	\draw [line width=2pt](4,0.1)--(4,-0.1) node [below]{$x+\varDelta x$};
	\draw [line width=2pt] (0.1,1)--(-0.1,1) node [left] {$f(x)$};
	\draw [line width=2pt] (0.1,2.5)--(-0.1, 2.5) node [left] {$f(x+\varDelta x)$};
	\draw (2.5,0)node[above=-1]{$\varDelta x$};
\end{tikzpicture}
$$f^\prime(x)=\lim_{\varDelta x\to a}\frac{f(x+\varDelta x)-f(x)}{\varDelta x}=\frac{MA}{M_0A}=\tg \alpha$$\\
При $M\to M_0\colon \alpha \to \varphi\Rightarrow f^\prime(x)=\tg\varphi$
\subsection{Механический смысл производной}
\begin{tikzpicture}

	\draw[|->, line width=1pt] node [above]{0} (0,0)--(7,0) node[right]{$t$};

	\draw [line width=2pt](3,0.1)--(3,-0.1) node [above=3]{$M$};
	\draw [line width=2pt](6,0.1) node[below] {$t+\varDelta t$}--(6,-0.1) node [above=3]{$M_1$};
	\draw[<->, line width=1pt] (0,-1)--(1.5,-1)node[above]{$S(t)$}--(3,-1);
	\draw [dashed] (0,0)--(0,-2);
	\draw [dashed] (3,0)--(3,-1);
	\draw [dashed] (6,0)--(6,-2);
	\draw[<->, line width=1pt] (0,-2)--(3,-2)node[above]{$S(t+\varDelta t)$}--(6,-2);
	
	\draw (4.5,0)node[below=-1]{$\varDelta t$};
	\draw (4.5,0)node[above=-1]{$\varDelta S$};

\end{tikzpicture}
$$V=\frac{\varDelta S}{\varDelta t}=\lim_{\varDelta t \to 0}\frac{S(t+\varDelta t) - S(t)}{\varDelta t} = S^\prime(t)$$
\subsection{Уравнение касательной к графику функции}
\begin{multline*}
	y_0=f(x_0)\\
	y-y_0=k(x-x_0)\\
	y=y_0+y^\prime(x_0)(x-x_0)\\
\end{multline*}
\section{Дифференцируемость функции в точке. Необходимое и достаточное условие дифференцируемости.  Непрерывность дифференцируемой функции.}
\subsection{Дифееренцируемость функции в точке}
$f(x)$ -- дифференцируема в точке $x$, если приращение функции $\varDelta f(x)$ можно представить в виде:
\begin{multline*}
\varDelta f(x)=A\cdot \varDelta x+\alpha (\varDelta x)\cdot \varDelta x \\
A\text{ -- число}\\
\alpha(\varDelta x) \text{ -- б.м. при }\varDelta x \to 0\\
\end{multline*}
\subsection{Необходимое и достаточное условие дифференцируемости.}
$f(x)$ -- дифференцируема в точке $x \Leftrightarrow \exists$ конечная $f^\prime(x)$
\textit{Доказательство:}
\begin{multline*}
	\boxed{\Leftarrow}\\
	f^\prime(x)=\lim_{\varDelta x \to 0}\frac{f(x+ \varDelta x)-f(x)}{\varDelta x}\\
	A=\lim f(x) \Leftrightarrow f(x)-A=\alpha(x)\\
	\Downarrow\\
	f^\prime(x)=\frac{\varDelta f(x)+\alpha_1(\varDelta x)}{\varDelta x}\\
	\Downarrow\\
	f^\prime(x)\varDelta x=\varDelta f(x)+\alpha_1(\varDelta x)\\
	\Downarrow\\
	\varDelta f(x)=f^\prime(x)\varDelta x+ \alpha(x)\varDelta x\\
	\alpha = -\alpha_1 \Rightarrow f(x) \text{ -- дифференцируема}\\
	\end{multline*}
	\begin{multline*}
		\boxed{\Rightarrow}\\
		f^\prime(x)=\lim_{\varDelta x \to 0}\frac{\varDelta f(x)}{\varDelta x}=\lim_{\varDelta x \to 0} \frac{A\varDelta x + \alpha(\varDelta x) \varDelta x}{\varDelta x}=\\
		=\lim_{\varDelta x \to 0} A + \lim_{\varDelta x \to 0} \alpha(\varDelta x)= A + 0 = A\\
		\blacksquare\\
	\end{multline*}
\subsection{Непрерывность дифференцируемой функции}
$f(x)$ -- дифференцируема в точке $x \Rightarrow f(x)$ непрерывна в точке $x$ 
\section{Дифференциал функции. Геометрический смысл дифференциала. Производная суммы, произведения и частного двух функций.}
$f^\prime(x)$ -- конечная $\Rightarrow \underbrace{f^\prime(x)\varDelta x} + \alpha(\varDelta x)\varDelta x$\\
$df(x)=f^\prime(x)\varDelta x$ -- дифференциал функции $f(x)$ в точке $x$.\\
$dx=\varDelta x$\\
$$f^\prime(x)=\frac{df(x)}{dx}$$
\subsection{Геометрический смысл дифференциала}
$df(x)=\tg\varphi\cdot\varDelta x$\\
\begin{tikzpicture}
	%	\draw[step=1.0,gray! 40,thin] (-5,-1) grid (5.5,4.5);
	\draw[->] (-5,0)--(5,0) node[right]{$x$};
	\draw[->] (0,-1)--(0,5) node[above]{$y$};
	
	\draw[line width=1pt] plot[domain=-1:5] (\x, {1+(\x - 1)*\x/(\x+4)});
	\draw[] plot[domain=-5:5] (\x, {(\x+4)/5});
	
	\draw (-2,0) arc (0:23.2:1);
	\draw (-1.5, 0)node[above=-1]{$\varphi$};
	%	
	\draw [dashed] (0,1)--(6.5,1);
	\draw [dashed] (0,2.5)--(6.5,2.5);
	\draw [dashed] (4,0)--(4,2.5) node [above] {$M$};
	\draw [dashed] (1,0)--(1,1) node[above]{$M_0$};
	\draw [dashed] (0,1.6)--(5,1.6);
	\draw [blue, line width=1pt](4,1) -- (4,2.5);
	\draw [red, line width=1pt](4,1) -- (4,1.6);

	\draw [<->, red, line width=1pt](4.5,1) --(4.5, 1.3) node [right]{$df(x)$} -- (4.5,1.6);
	\draw [<->, blue, line width=1pt](6,1) --(6, 1.75) node [right]{$\varDelta f(x)$} -- (6,2.5);
	\draw [line width=2pt](1,0.1)--(1,-0.1) node [below=3]{$x$};
	\draw [line width=2pt](4,0.1)--(4,-0.1) node [below]{$x+\varDelta x$};
	\draw [line width=2pt] (0.1,1)--(-0.1,1) node [left] {$f(x)$};
	\draw [line width=2pt] (0.1,2.5)--(-0.1, 2.5) node [left] {$f(x+\varDelta x)$};
	\draw (2.5,0)node[above=-1]{$\varDelta x$};
\end{tikzpicture}
\subsection{Производная суммы}
\begin{align*}
	u&=u(x); v = v(x)\\\\
	y^\prime &= \lim_{\varDelta x \to 0}\frac{\left(u(x+\varDelta x)\pm v(x+\varDelta x)\right) - \left(u(x)\pm v(x)\right)}{\varDelta x}=\\
	&=\lim_{\varDelta x \to 0}\left(\frac{u(x+\varDelta x) - u(x)}{\varDelta x}\right)\pm \lim_{\varDelta x \to 0}\left(\frac{v(x+\varDelta x) - v(x)}{\varDelta x}\right)=\\
	&= u^\prime(x)\pm v^\prime(x)\\
\end{align*}
\subsection{Производная произведения}
\begin{align*}
	u&=u(x); v = v(x)\\\\
	y^\prime &= \lim_{\varDelta x \to 0}\frac{\left(u(x+\varDelta x)\cdot v(x+\varDelta x)\right) - \left(u(x)\cdot v(x)\right)}{\varDelta x}=\\
	&=\lim_{\varDelta x \to 0}\frac{\left(u(x)+\varDelta u\right) \cdot \left(v(x + \varDelta v)\right) - u(x)\cdot v(x)}{\varDelta x}=\\
	&=\lim_{\varDelta x \to 0}\frac{\cancel{u(x)v(x)}+u(x)\varDelta v+ v(x)\varDelta u +\varDelta u \varDelta v -\cancel{u(x) v(x)}}{\varDelta x}=\\
	&=\lim_{\varDelta x \to 0}\left(u(x)\cdot\frac{\varDelta v}{\varDelta x}+ v(x)\frac{\varDelta u}{\varDelta x} +\frac{\varDelta u \varDelta v}{\varDelta x}\right)=\\
	&=u(x)\lim_{\varDelta x \to 0}\frac{\varDelta v}{\varDelta x}+v(x)\lim_{\varDelta x \to 0}\frac{\varDelta u}{\varDelta x}+\cancelto{0}{\lim_{\varDelta x \to 0}\varDelta u}\cdot \lim_{\varDelta x \to 0}\frac{\varDelta v}{\varDelta x}=\\
	&=u(x)v^\prime(x)+u^\prime(x)v(x)
\end{align*}
\subsection{Производная частного}
\begin{align*}
	u&=u(x); v = v(x)\\\\
	y^\prime &= \lim_{\varDelta x \to 0} \frac
	{\frac
		{u(x+\varDelta x)}
		{v(x+\varDelta x)}
		 - 
		 \frac
		 {u(x)}
		 {v(x)}}
	{\varDelta x}=
	\lim_{\varDelta x \to 0} \frac
	{\frac
		{u(x)+\varDelta u}
		{v(x)+\varDelta v}
		- 
		\frac
		{u(x)}
		{v(x)}}
	{\varDelta x}=\\
	&= \lim_{\varDelta x \to 0} \frac
	{
		\cancel{u(x)v(x)}+v(x)\varDelta u-\cancel{u(x)v(x)}-u(x)\varDelta v
	}
	{\varDelta x v(x)\left(v(x)+\varDelta v\right)}=\\
	&= \lim_{\varDelta x \to 0} \frac
	{
		v(x)\varDelta u-u(x)\varDelta v
	}
	{\varDelta x \left(v^2(x)+\varDelta vv(x)\right)}=\\
	&= \lim_{\varDelta x \to 0} \frac
	{
		\cancel{u(x)v(x)}+v(x)\varDelta u-\cancel{u(x)v(x)}-u(x)\varDelta v
	}
	{\varDelta x v(x)\left(v(x)+\varDelta v\right)}=\\
	&= \lim_{\varDelta x \to 0} \frac{
		v(x)\frac
		{\varDelta u}{\varDelta x}-u(x)\frac{\varDelta v}{\varDelta x}
	}
	{v^2(x)+\varDelta vv(x)}=\\
	&=\frac{\begin{gathered}
		v(x)\cdot \lim_{\varDelta x \to 0}\frac{\varDelta u}{\varDelta x} - u(x)\cdot \lim_{\varDelta x \to 0}\frac{\varDelta v}{\varDelta x}\end{gathered}}{\begin{gathered}
		v^2+v\cdot\lim_{\varDelta x \to 0}\varDelta u \end{gathered}}=\\
	&=\frac{v(x)u^\prime(x)-u(x)v^\prime(x)}{v^2(x)}\\
\end{align*}
\section{Теорема о дифференцируемости сложной функции. }
\begin{multline*}
	\left .
	\begin{gathered}
		\begin{aligned}
			&u(x)\text{ -- дифференцируема в точке }x\\
			&f(u)\text{ -- дифференцируема в точке }u=u(x)\\
		\end{aligned}
	\end{gathered}
	\right\}\Rightarrow f(u(x))\text{ -- дифференцируема в точке } x\\
	\frac{df}{dx}=f_u^\prime\left(u(x)\right)\cdot u^\prime(x)\\	
\end{multline*}
\textit{Доказательство:}
\begin{align*}
	\varDelta f&=f\left(u+\varDelta u\right)-f(u)=f^\prime(u)\varDelta u + \alpha(\varDelta u)\cdot\varDelta u\\
	f^\prime(x)&=\lim_{\varDelta x \to 0}\frac{f^\prime(u)\varDelta u + \alpha(\varDelta u)\cdot \varDelta u}{\varDelta x}=\\
	&=\lim_{\varDelta x \to 0}\frac{f^\prime(u)\varDelta u}{\varDelta x} + \cancelto{0}{\lim_{\varDelta x \to 0}\frac{\alpha(\varDelta u)\cdot \varDelta u}{\varDelta x}}=\\
	&=f^\prime(u)\lim_{\varDelta x \to 0}\frac{\varDelta u}{\varDelta x}=f^\prime(u)u^\prime(x)\\
	\blacksquare\\
\end{align*}
\section{Производная обратной функции. Вывод производных обратных тригонометрических функций. }
\subsection{Производная обратной функции}
$y=f(x)$ -- монотонна в $U(x)$ и дифференцируема в точке $x, f^\prime(x)\neq0$\\
$g(y)$ -- обратная для $f(x)$\\
$\Rightarrow g(y)$ -- дифференцируема в точке $y=y(x)$ и $$g^\prime(y)=\frac{1}{f^\prime(x)}$$
\textit{Доказательство:}
\begin{align*}
	g^\prime(y)&=\lim_{\varDelta y \to 0}\frac{g(y+\varDelta y)-g(y)}{\varDelta y}=\lim_{\varDelta x \to 0}\frac{(x+\varDelta x) - x}{f(x+\varDelta x)-f(x)}=\\
	&=\lim_{\varDelta x \to 0}\frac{1}{\frac{f(x+\varDelta x)-f(x)}{(x+\varDelta x) - x}}=\frac{1}{f^\prime(x)}\\
	\blacksquare\\
\end{align*}
\subsection{Вывод обратных тригонометрических функций}
\begin{align*}
	&f(x)=\sin x, &x\in\left[-\frac{\pi}{2};\frac{\pi}{2}\right]\\
	&g(x)=\arcsin y\\
	&g^\prime(y)=\frac{1}{cos x}=\frac{1}{\sqrt{1-y^2}}\\
	&\left(\arcsin x\right)^\prime = \frac{1}{\sqrt{1-x^2}}\\
	&\left(\arccos x\right)^\prime = -\frac{1}{\sqrt{1-x^2}}\\
	&\left(\arctg x\right)^\prime = \frac{1}{1+x^2}\\
	&\left(\arcctg x\right)^\prime = -\frac{1}{1+x^2}\\
\end{align*}
\section{Правило логарифмического дифференцирования. Его применение к нахождению производных функций  $f(x)=a^x$,   $f(x)=x^a$.}
\begin{multline*}
	y=f^\prime(x)\\
	\left(\ln f(x)\right)^\prime = \frac{f^\prime (x)}{f(x)}\\
\end{multline*}
\begin{enumerate}
	\item $f(x)=a^x$
	\begin{multline*}
		\ln f(x)=x \ln a\\
		\frac{f^\prime(x)}{f(x)}=\ln a\\
		f^\prime(x)=\ln a\cdot f(x)= \ln a\cdot a^x\\
	\end{multline*}
	\item $f(x)=x^a$
	\begin{multline*}
		\ln f(x) = a \ln x\\
		\frac{ f^\prime(x)}{f(x)}=\ln a\\
		f^\prime(x)=\frac{a}{x}\cdot x^a\\
		f^\prime(x) = a\cdot x^{a-1}\\
	\end{multline*}
\end{enumerate}
\section{Производные и дифференциалы высших порядков.}
\begin{multline*}
	y=f(x)\\
	\exists f^\prime(x) \text{ -- первого порядка}\\
	\left(f^\prime(x)\right)^\prime \text{ -- производная второго порядка}\\
	\text{\textit{Обозначение}:}\\
		f^{\prime\prime}(x) = \frac{d^2f}{dx^2}\\
		f^{\prime\prime\prime}(x)= \frac{d^3f}{dx^3}\\
		f^{(4)}(x)= \frac{d^4f}{dx^4}\\
		f^{(n)}(x)= \frac{d^nf}{dx^n}\\
\end{multline*}
\textit{Дифференциал:}
\begin{multline*}
	df(x)=f^\prime(x)dx\\
	\text{2-го порядка}:\\
	d(df(x))=d(f^\prime(x)dx)=\left(f^\prime(x)d(x)\right)^\prime dx=f^{\prime\prime}(x)dx^2\\
	\text{3-го порядка}:\\
	f^{\prime\prime\prime}(x)dx^3\\
	\text{4-го порядка}:\\
	f^{(4)}(x)dx^4\\
	\text{n-го порядка}:\\
	f^{(n)}(x)dx^n\\
\end{multline*}
\section{Таблица производных.}
\includegraphics[width=1\linewidth]{1298487_1.jpeg}
\section{Дифференцирование функций, заданных параметрически (первая и вторая производные). Теорема Ролля, ее геометрический смысл.}
\subsection{Дифференцирование функций, заданных параметрически}
\begin{multline*}
	\begin{cases}
		x=\varphi(t) \text{ -- дифференцируема, монотонна}\\
		y=\psi(t)\text{ -- дифференцируема}\\
	\end{cases}\\
	\Downarrow\\
	\exists \varphi^{-1}(x)\Rightarrow y(x)=\psi\left(\varphi^{-1}(x)\right)\\
	\text{Обозначим } g(x)=\varphi^{-1}(x)\\
	g(x) \text{ -- обратная для }\varphi(x)\Rightarrow g^\prime(x)=\frac{1}{\varphi^\prime(t)}\\
	y(x)= \psi(g(x))\\
	y^\prime(x)=\psi^\prime(x)\cdot g^\prime(x) = \psi^\prime(t)\cdot \frac{1}{\varphi^\prime(t)}=\frac{\psi^\prime(t)}{\varphi^\prime(t)}\\
	y^{\prime\prime}(x)=\frac{\left(\frac{\psi^\prime(t)}{\varphi^\prime(t)}\right)^\prime}{\varphi^\prime(t)}\\
\end{multline*}
\subsection{Теорема Ролля}
$f(x)$ -- непрерывна на $[a;b]$, дифференцируема на $(a;b)$\\
$f(a)=f(b) \Rightarrow \exists$ хотя бы одна $c \in (a; b)\colon f^\prime(c)=0$\\
\textit{Доказательство:}
\begin{enumerate}
	\item $f(x) = const \Rightarrow c $ -- $\forall$ точка $\in (a; b)$
	\item $f(x) \neq const \Rightarrow $ по Теореме Ферма.
\end{enumerate}
\begin{tikzpicture}
	%	\draw[step=1.0,gray! 40,thin] (-5,-1) grid (5.5,4.5);
	\draw[->] (-1,0)--(4,0) node[right]{$x$};
	\draw[->] (0,-1)--(0,4) node[above]{$y$};

	\draw[line width=1pt] plot[domain=1:3] (\x, {2-(\x-2)*(\x-2)});
	
%	\draw (-2,0) arc (0:23.2:1);
%	\draw (-1.5, 0)node[above=-1]{$\varphi$};
	%	
	\draw [dashed] (0,1)--(3,1);
	\draw [dashed] (1,0)--(1,1);
	\draw [dashed] (3,0)--(3,1);
	\draw [dashed] (2,0)--(2,2);
	\draw [line width=2pt](1,0.1)--(1,-0.1) node [below=3]{$a$};
	\draw [line width=2pt](3,0.1)--(3,-0.1) node [below=]{$b$};
	\draw [line width=2pt](2,0.1)--(2,-0.1) node [below=3]{$c$};
	\fill (1, 1) circle (2pt);
	\fill (3, 1) circle (2pt);
	\fill (2, 2) circle (2pt);
	%	\draw [line width=2pt](4,0.1)--(4,-0.1) node [below]{$x+\varDelta x$};
%	\draw [dashed] (0,2.5)--(6.5,2.5);
%	\draw [dashed] (4,0)--(4,2.5) node [above] {$M$};
%	\draw [dashed] (1,0)--(1,1) node[above]{$M_0$};
%	\draw [dashed] (0,1.6)--(5,1.6);
%	\draw [blue, line width=1pt](4,1) -- (4,2.5);
%	\draw [red, line width=1pt](4,1) -- (4,1.6);
%	
%	\draw [<->, red, line width=1pt](4.5,1) --(4.5, 1.3) node [right]{$df(x)$} -- (4.5,1.6);
%	\draw [<->, blue, line width=1pt](6,1) --(6, 1.75) node [right]{$\varDelta f(x)$} -- (6,2.5);
%	\draw [line width=2pt](1,0.1)--(1,-0.1) node [below=3]{$x$};
%	\draw [line width=2pt](4,0.1)--(4,-0.1) node [below]{$x+\varDelta x$};
%	\draw [line width=2pt] (0.1,1)--(-0.1,1) node [left] {$f(x)$};
%	\draw [line width=2pt] (0.1,2.5)--(-0.1, 2.5) node [left] {$f(x+\varDelta x)$};
%	\draw (2.5,0)node[above=-1]{$\varDelta x$};
\end{tikzpicture}
\section{Теорема Коши. Формула конечных приращений Лагранжа, ее геометрический смысл.}
\subsection{Теорема Коши}
$f(x), g(x)$ -- непрерывны на $\left[a;b\right]$, дифференцируемы на $\left(a;b\right)$\\
$g(x)\neq 0, x \in \left(a;b\right) \Rightarrow \exists$ хотя бы одна точка $c\in\left(a;b\right)$:\\
$$\frac{f(b)-f(a)}{g(b)-g(a)}=\frac{f^\prime(c)}{g^\prime(c)}$$
\textit{Доказательство:}
\begin{multline*}
	F(x)=f(x) - R\left(g(x)-g(a)\right)\\
	F(a)=f(a)\\
	F(b)= f(b)\\
	\text{По Теореме Ролля: } \exists c \colon f^\prime(c)=0\\
	F^\prime(c)=f^\prime(c)-R\left(g^\prime(c)-0\right)=0\\
	\Downarrow\\
	R=\frac{f^\prime(c)}{g^\prime(c)}\\
	\blacksquare\\
\end{multline*}
\subsection{Формула конечных приращений Лагранжа}
\begin{multline*}
	a=x\\
	b=x+\varDelta x\\
	f(b)-f(a)=f^\prime(c)(b-a)\\
	f(x+\varDelta x)-f(x)=f^\prime(c)\cdot \varDelta x\\
	f(b)-f(a)=BC\\
	b-a=AC\\
	\frac{f(b)-f(a)}{b-a}=f^\prime(c)=\tg \angle BAC\\
\end{multline*}
\begin{tikzpicture}
	%	\draw[step=1.0,gray! 40,thin] (-5,-1) grid (5.5,4.5);
	\draw[->] (-1,0)--(5,0) node[right]{$x$};
	\draw[->] (0,-1)--(0,5) node[above]{$y$};
	
	
	\draw[line width=1pt] plot[domain=1:3] (\x, {1+sqrt(9-((\x-4)*(\x-4))});	
%	\draw[] plot[domain=-2:3] (\x, {1+(2*\x+1)/sqrt(5)});	
	\draw[] plot[domain=-1:3] (\x, {1.194*\x+0.896});	
	
		\draw (-0.2,0) arc (0:30:1);
%		\draw (0, 0)node[above=-1]{$\varphi$};
	%	
	\draw [dashed] (0,1)--(3,1);
	\draw [dashed] (1,0)--(1,1);
	\draw [dashed] (3,0)--(3,4);
	\draw [dashed] (1.7,0)--(1.7,2.92);
	\draw [line width=2pt](1,0.1)--(1,-0.1) node [below=3]{$a$};
	\draw [line width=2pt](3,0.1)--(3,-0.1) node [below=]{$b$};
	\draw [line width=2pt](1.7,0.1)--(1.7,-0.1) node [below=3]{$c$};
	\fill (1, 1) circle (2pt) node[anchor=south west]{$A$};
	\fill (3, 3.85) circle (2pt)node[anchor=north west]{$B$};
	\fill (3, 1) circle (2pt)node[anchor=south west]{$C$};
	\fill (1.7, 2.92) circle (2pt);
	%	\draw [line width=2pt](4,0.1)--(4,-0.1) node [below]{$x+\varDelta x$};
		\draw [dashed] (0,3.85)--(3,3.85);
	%	\draw [dashed] (4,0)--(4,2.5) node [above] {$M$};
	%	\draw [dashed] (1,0)--(1,1) node[above]{$M_0$};
	%	\draw [dashed] (0,1.6)--(5,1.6);
	%	\draw [blue, line width=1pt](4,1) -- (4,2.5);
	%	\draw [red, line width=1pt](4,1) -- (4,1.6);
	%	
	%	\draw [<->, red, line width=1pt](4.5,1) --(4.5, 1.3) node [right]{$df(x)$} -- (4.5,1.6);
	%	\draw [<->, blue, line width=1pt](6,1) --(6, 1.75) node [right]{$\varDelta f(x)$} -- (6,2.5);
	%	\draw [line width=2pt](1,0.1)--(1,-0.1) node [below=3]{$x$};
	%	\draw [line width=2pt](4,0.1)--(4,-0.1) node [below]{$x+\varDelta x$};
		\draw [line width=2pt] (0.1,1)--(-0.1,1) node [left] {$f(a)$};
		\draw [line width=2pt] (0.1,3.85)--(-0.1, 3.85) node [left] {$f(b)$};
	%	\draw (2.5,0)node[above=-1]{$\varDelta x$};
\end{tikzpicture}
\section{Правило Лопиталя.}
Для неопределенностей $\left[\frac{0}{0}\right]$ и $\left[\frac{\infty}{\infty}\right]$
\paragraph{Теорема 1}
$f(x), g(x)$ -- дифференцируемы в $U(a)$ и бесконечно малые в точке $a, g(a)\neq0\Rightarrow$
$$\lim_{x\to a} \frac{f(x)}{g(x)}=\lim_{x \to a}\frac{f^\prime(x)}{g^\prime(x)} \text{ если }\exists$$
\textit{Доказательство:}
\begin{multline*}
	\frac{f(x)}{g(x)}=\frac{f(x) -f(a)}{g(x)-g(a)}=\text{(Теорема Коши)}=\frac{f^\prime(c)}{g^\prime(c)}=\lim_{x\to a}\frac{f^\prime(x)}{g^\prime(x)}
\end{multline*}
\paragraph{Теорема 2}
$f(x), g(x)$ -- бесконечно большие в точке $a\Rightarrow$
$$\lim_{x\to a}\frac{f(x)}{g(x)}=\lim_{x \to a}\frac{f^\prime(x)}{g^\prime(x)}$$ 
\section{Формула Тейлора для функции одной переменной с остаточным членом в форме Лагранжа и форме Пеано. }
$f(x)$ -- определена в $U(x_0)$ и имеет в ней производные до $(n+1)$ порядка включительно, то для любого $x$ из этой окрестности найдется $c\in\left(x_0;x\right)\colon$
$$f(x)=f(x_0)+\frac{f^\prime(x_0)}{1!}\left(x-x_0\right)+\frac{f^{\prime\prime}(x_0)}{2!}\left(x-x_0\right)^2+\dots+\frac{f^{(n)}(x_0)}{n!}\left(x-x_0\right)^n+\frac{f^{(n+1)}(c)}{(n+1)!}\left(x-x_0\right)^{n+1}$$
Последний член -- остаточный.\\
В форме Лагранжа: $$\begin{gathered}
	\frac{f^{(n+1)}(c)}{(n+1)!}\left(x-x_0\right)^{n+1}
\end{gathered}$$
В форме Пеано:
$$\begin{gathered}
	\frac{f^{(n+1)}(x_0)}{(n+1)!}\left(x-x_0\right)^{n+1}
\end{gathered}, c\in (x;x_0)$$
\section{Формулы Тейлора (Маклорена) для функций $y = e^x$,  $y = \sin x$, $y = \cos x$, ,   в окрестности точки x = 0.}
При $x_0=0$
\begin{multline*}
	f(x)=f(0)+\frac{f^\prime(0)}{1!}x+\frac{f^{\prime\prime}(0)}{2!}x^2+\dots+\frac{f^{(n)}(0)}{n!}x^n+\frac{f^{(n+1)}(0)}{(n+1)!}x^{n+1}\\
\end{multline*}
\begin{enumerate}
	\item $e^x$
	\begin{align*}
		f(x) &= e^x&&f(0)=1\\
		f^\prime(x)&=e^x&&f^\prime(0)=1\\
		f^{\prime\prime}(x)&=e^x&&f^{\prime\prime}(0)=1\\
		&\dots&&\dots\\
		f^{(n)}(x)&=e^x&&f^{(n)}(0)=1\\
	\end{align*}
	$$e^x=1+\frac{1}{1!}x+\frac{1}{2!}x^2+\dots+\frac{1}{n!}x^n+\frac{e^0}{(n+1)!}\\$$
	\item $\sin x$
	\begin{align*}
		f(x) &= \sin x&&f(0)=0\\
		f^\prime(x)&=\cos x&&f^\prime(0)=1\\
		f^{\prime\prime}(x)&=-\sin(x)&&f^{\prime\prime}(0)=0\\
		f^{\prime\prime\prime}(x)&=-\cos(x)&&f^{\prime\prime}(0)=-1\\
		f^{(4)}(x)&=\sin(x)&&f^{\prime\prime}(0)=0\\
	\end{align*}
	$$\sin x=0+\frac{1}{1!}+\frac{0}{2!}-1\frac{1}{3!}+\frac{0}{4!}+\frac{1}{5!}+\dots+\left(-1\right)^n\frac{x^{2n+1}}{(2n+1)!}$$
\end{enumerate}
\section{Необходимое и достаточное условия возрастания (убывания) функции  y = f (x).}
\subsection{Необходимое условие}
$f(x)$ -- дифференцируема на $X$\\
\begin{multline*}
	f(x) \nearrow \Rightarrow f^\prime(x)\geq 0, x \in X\\
	f(x) \searrow \Rightarrow f^\prime(x)\leq 0, x \in X\\
\end{multline*}
\subsection{Достаточное условие}
\begin{multline*}
	f^\prime(x)>0, x \in X \Rightarrow f(x)\nearrow\\
	f^\prime(x)<0, x\in X \Rightarrow f(x) \searrow\\
\end{multline*}
\section{Определение экстремума функции $y = f (x)$. Необходимое условие экстремума.}
$f(x)$ -- определена на $X$\\
$x_0$ -- $max$, если $\exists U(x_0)\colon f(x)<f(x_0)\forall x \in U(x_0)$\\
$x_0$ -- $min$, если $\exists U(x_0)\colon f(x)>f(x_0)\forall x \in U(x_0)$\\\\
$f(x)$ -- дифференцируема на $X, x_0 \in X$\\
$x_0$ -- точка экстремума $\Rightarrow f^\prime(x_0)=0$
\section{Достаточное условие экстремума, использующее первую производную.}
$f(x)$ -- дифференцируема на $X$\\
$x_0$ -- бесконечно малая точка, определенная на $X$
\begin{multline*}
	\left.
	\begin{gathered}
		f^\prime(x)>0 \text{ при } x<x_0\\
		f^\prime(x)<0 \text{ при } x>x_0\\
	\end{gathered}
	\right\}\Rightarrow x_0 - \max\\
	\left.
	\begin{gathered}
		f^\prime(x)<0 \text{ при } x<x_0\\
		f^\prime(x)>0 \text{ при } x>x_0\\
	\end{gathered}
	\right\}\Rightarrow x_0 - \min\\
	f^\prime(x)=0 \text{ или } \infty  \text{ или } \nexists \text{ -- критическая точка (подозрительная на экстремум)}\\
	f^\prime(x)>0 \text{ -- стационарная точка}\\
\end{multline*}
\section{Достаточное условие экстремума, использующее вторую производную.}
$f(x)$ -- два раза непрерывна на $X$, дифференцируема на $X$\\
$x=x_0$ -- стационарная точка $(f^\prime(x)=0)$\\
$\Rightarrow f^{\prime\prime} < 0\Rightarrow x_0 - \max$\\
$\Rightarrow f^{\prime\prime}> 0\Rightarrow x_0 - \min$\\
\section{Определение направления выпуклости графика функции  $y = f (x)$. Признак выпуклости вверх и выпуклости вниз. Точки перегиба графика функции.}
Выпуклая вверх на $X \Leftrightarrow \forall x, x+\varDelta x \in X \Rightarrow\varDelta y<dy$\\
Выпуклая вниз на $X \Leftrightarrow \forall x, x+\varDelta x \in X \Rightarrow\varDelta y>dy$\\
\subsection{Достаточный признак направления выпуклости}
$f(x)$-- 2 раза дифференцируема на $X$\\
$\Rightarrow$ Если $f^{\prime\prime}(x)>0$ на $X \Rightarrow$ график выпуклый вниз на $X$\\
$\Rightarrow$ Если $f^{\prime\prime}(x)<0$ на $X \Rightarrow$ график выпуклый вверх на $X$\\
Точки, в которых меняется направление выпуклости -- точки перегиба
\section{Асимптоты графика функции  $y = f (x)$. Правило нахождения вертикальных и невертикальных асимптот.}
\textbf{\textit{Асимптота}} -- прямая, к которой приближается график функции (расстояние $\to 0 $)
\begin{enumerate}
	\item Вертикальная асимптота.
	$x=a$ -- вертикальная асимптота, если:
	\begin{equation*}
		\begin{cases}
			a \text{ -- точка бесконечного разрыва}\\
			a \text{ -- граничная точка области определения, если односторонний предел } = \infty\\
		\end{cases}
	\end{equation*}
	\item Наклонная асимптота
	\begin{multline*}
		y=kx+b\\
		k=\lim_{x\to \pm \infty}\frac{f(x)}{x}\\
		b = \lim_{x\to \pm \infty}f(x)-kx\\
	\end{multline*}
Если хотя бы один из этих пределов $\nexists$ или $\infty$, то наклонных асимптот нет.
\end{enumerate}
\section{Определение первообразной. Теорема о двух первообразных одной функции.}
$F(x), f(x)$ -- определена на $X$\\
$F(x)$ -- первообразная для $f(x)$, если $F^\prime(x)=f(x)$
\paragraph{Теорема} $F_1(x)$ и $F_2(x)$ -- первообразные для $f(x)$, то $F_1(x)=F_2(x)+c, c - const$\\
\textit{Доказательство:}
\begin{multline*}
	\left(f_1(x)-F_2(x)\right)^\prime=f(x)-f(x)=0 \Rightarrow F_1(x)-F_2(x)= c\\
\end{multline*}
\section{Определение неопределенного интеграла и его свойства. Инвариантность формул интегрирования.}
\section{Метод интегрирования по частям для неопределенного интеграла.}
\section{Метод подстановки для неопределенного интеграла.}
\section{Простейшие рациональные дроби и их интегрирование.}
\section{Интегрирование дробно-рациональных функций.}
\section{Интегрирование тригонометрических функций.}
\section{Использование подстановок ,  , примеры.}
\end{document}
